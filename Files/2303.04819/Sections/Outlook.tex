\startdocument

\section{Outlook}

  Observations in the last few decades  have completely transformed our view of the universe. They  have also raised puzzles on 
  all length scales: What drives acceleration of the present universe? Why are there super-horizon correlations in the CMB? What is dark matter?  The standard model of cosmology, together with inflation, provides a paradigm to accommodate these issues, but does not explain them from a microscopic theory and clearly needs to be put on a more solid theoretical footing. Some of
 these puzzles may have their roots in quantum gravity, and hence cosmology and quantum gravity need to be brought together. 
 
 String theory's remarkable mathematical
 structure has repeatedly shown that it contains all the ingredients needed for a quantum theory of gravity --  ultraviolet finiteness, 
 an understanding of black hole entropy  and explicit realisations of holography. However, string theory is not simply a theory of quantum gravity alone: it also comes with particles
 and interactions. This unison gives it the necessary elements to attack the present day questions in cosmology.  This review
is evidence of the tremendous progress that has been made over the years. At the same time, there remain many outstanding challenges; given the importance of these open questions, string cosmology requires further intense pursuit. 
 
 In terms of formal aspects,  understanding cosmological singularities remains the main open question. The level of progress made on other singular geometries (black holes), both in terms of matching microscopic counts of degrees of freedom with the entropy and also in terms of more quantitative understandings of how information is preserved in the process of Hawking radiation, makes us  optimistic that in the not too far future, a proper understanding of cosmological big-bang singularities may be achieved. We can hope for the same regarding the status of explicit realisations of de Sitter space within string theory. 
 
 It is often stated that string theory is decoupled from observational physics. Indeed, it is not possible to single out an in-out prediction accessible to current technology  that could rule out the theory. On one hand, there are strong indications that the theory is unique. On the other hand, there is an embarrassment of richness  once four-dimensional solutions are considered. Various ingredients (the possibly infinite diversity of compactification manifolds with different curvatures,  the huge number of quantised fluxes for each of these manifolds, the different brane configurations, the quantum corrections to leading order calculations, the possible non-critical solutions) all lead to a large multitude of vacua.
 And yet, despite many efforts, there is not a single string model that can be called fully realistic. Current experimental constraints are already enough to rule out most constructions, either through tests of fifth forces, Standard Model matter constraints, or the existence of relics or exotic particles.
 
 Extracting model independent properties has always been one of the main goals in the effort to confront string theory with data. Particle physics constraints are very important but are intrinsically model dependent as they depend on the
 nature of gauge symmetries, chiral matter content and couplings within a model. Cosmology offers greater chances for universality -- although some aspects are definitely model-dependent, as illustrated by the number of possible string candidates for inflation discussed in chapter 4. It is important to emphasise that each one of these represents a large class of models, as they refer to the nature of the corresponding inflaton candidate, and, in the best scientific tradition, they make concrete predictions that can be compared with experiments. Of the dozen or so scenarios listed, several of them are already in tension with data. This is a reminder that even if data cannot rule out string theory as such, the more standard and less ambitious programme of testing general classes of models is already under way. 
 
At this point we should emphasise that even though many string models of inflation have been constructed, none of them are particularly compelling. Much work therefore remains to be done regarding explicit realisations,  including moduli stabilisation and a potential for realistic matter. This task is closely related to the existence of de Sitter solutions in string theory, which also require careful accounts of moduli stabilisation. Furthermore, it is worth remarking that most alternative scenarios to string inflation include some period of contraction, although a contracting epoch does not exclude a subsequent period of inflation. 

One generic aspect of string theory is the existence of many possible low-energy solutions (often referred to as the string landscape). Associated to the string landscape is the controversial claim that this multiplicity of vacua, together with anthropic arguments, can be used to address the smallness of the cosmological constant, through the population of a vast discretuum of vacua.  Even leaving aside the controversial and unappealing aspects of anthropic arguments, before this approach can be claimed as successful it must also address important issues such as the population of the landscape and the measure problem. It has also been claimed that if our universe is the outcome of a vacuum transition from another universe as predicted by the landscape,  our universe should be that of an open universe.\footnote{This claim has been questioned recently \cite{Hawking:2017wrd,Cespedes:2020xpn}  but more work needs to be done before arriving to a definitive conclusion.} Even though current observations are consistent with a flat universe it may be possible that in not too far a future a non-zero curvature of the universe is determined. If this turns out to be closed rather than open, this would rule out this landscape scenario (if the open universe claim is correct).

One more concrete, but still generic, prediction from string theory is the existence of extra dimensions and the associated moduli fields. These may be heavy if compactifications have no residual supersymmetry, but in supersymmetric compactifications the moduli are typically light and their existence can imply substantial modifications to the epoch between inflation and BBN. We have discussed many implications of moduli, such as
 kination epochs, moduli domination, moduli reheating, dark radiation, and the possible existence of oscillons or oscillatons. We have also discussed possible ways to subject these ideas to potential experimental tests.

One appealing aspect of the swampland and bootstrap programmes is their model independent nature, as they aim to provide general 
constraints, not only on what is achievable within string theory but also on what can hold in any proposed theory of quantum gravity. As we discussed in the previous section, at the moment the associated constraints are limited.  The swampland conjectures on more solid footings 
(such as the absence of global symmetries or the weak gravity conjectures) are the ones with the least phenomenological or cosmological impact, while those with major observational consequences (like the de Sitter or trans-Planckian conjectures) are more speculative. However, even setting aside their implications for potential observations, conceptually these conjectures may help to shape and understand the underlying quantum theory of gravity, string theory or otherwise.  Progress in streamlining these conjectures and constructing
explicit string models in accordance with them (such as those of quintessence) would be most welcome.

For potential observations, just as in particle physics, any single experimental test of a string scenario can probably also be described within a standard effective field theory independent of string theory (as an example, moduli can always be described in quantum field theory simply as scalar particles with non-renormalisable interactions). The best we can expect at this stage is a correlation of different observations of physics beyond the Standard Model. For example, a string model has to explain not only the early universe but also BSM physics. Furthermore, string models of inflation do not end simply with the end of inflation but contain a richness of physics after inflation due to the moduli fields.   
  
 Finally, the detection of gravitational waves has opened a new era in cosmology. Just
 as in  the of case electromagnetic waves, we will eventually study the universe with gravitational waves across a wide range of frequencies.  Hopefully, these truly gravitational probes will give us the hints we need to make the relation between strings and cosmology a quantitative one.
 
 We finish with a quote of Steven Weinberg that is apt for string cosmology: 
 
 \begin{quote}{\it The test of a physical theory is not that everything in it should be observable and every prediction it makes should be testable, but rather that enough is observable and enough predictions are testable to give us confidence that the theory is right.}
 \end{quote} %$\&$ {\it ``Our mistake is not that we take our theories too seriously, but that we do not take them seriously enough."}
   
We hope that this review is found useful in order to identify the main achievements and challenges in this exciting field. Let us also be optimistic
-- and hope that this review is soon rendered out of date by new discoveries and new observations. 
 
 \section{Acknowledgements}
 We thank all of our colleagues  that have shaped our understanding of this field and especially our collaborators on the subjects related to this review, including Shehu Abdussalam, Steve Abel,  Bobby Acharya, Yashar Aghababaie, Gerardo Aldazabal, Rouzbeh Allahverdi, Stephen Angus, Stefan Antusch, Luis Aparicio, Fien Apers, Fabio Apruzzi, Vikas Aragam, Tassos Avgoustidis,
 Vijay Balasubramanian, Arka Banerjee, Luke Barclay, Bruno V.~Bento, Marcus Berg, Per Berglund, Sukannya Bhattacharya,  Johan Bl{\aa}b\"ack, Jose Blanco-Pillado, Ralph Blumenhagen, Andrea Borghese, Philippe Brax, Igor Broeckel,   Cliff Burgess, Nana Cabo-Bizet, Philip Candelas, Alberto Casas, Juan Cascales, Pablo Camara, Francesco Cefala, Sebastian Cespedes, Sarah Chadburn, Dibya Chakraborty, Athanasios Chatzistavrakidis, Roberta Chiovoloni, Debika Chowdhury, David Ciupke, Jim Cline, Katy Clough, Daniel Cremades, Niccol\'o Cribiori, Chiara Crino, Francesc Cunillera, Kumar Das, Subinoy Das, Francesca Day,
  Shanta de Alwis,  Beatriz de Carlos, Xenia de la Ossa, Jean-Pierre Derendinger, Mansi Dhuria, Cristina Diamantini, Victor Diaz, Giuseppe Dibitetto, Kostas Dimopoulos, Eleonora Di Valentino,  Danielle Dineen, Matthew Dolan, Sean D. Downes, Bhaskar Dutta, Koushik Dutta, Damien A.~Easson,  Encieh Erfani, Cristina Escoda, Hassan Firouzjahi, Anamaria Font, Gabriele Franciolini, Mayukh Gangopadhyay, Juan Garc\'a-Bellido, Inaki Garc\'ia Etxebarria, Maria Pilar Garc\'ia del Moral, Fridrik Gautason, Ghazal Geshnizjani, Joaquim Gomes, Nicolas Grandi, Christophe Grojean, Andrew Frey, Diptimoy Ghosh, Steve Giddings, Marta G\'omez-Reino, Mark Goodsell, Ruth Gregory, Veronica Guidetti, Rajesh Gupta, Ulrich Haisch, Ed Hardy, Ehsan Hatefi,  Arthur Hebecker, Johnny Holland, Tristan Hubsch, Janet Hung, Luis Iba\~nez, Joerg Jaeckel, Nicholas Jennings, Esteban Jimenez, Renata Kallosh, Denis Klevers, Tatsuo Kobayashi, Steve Kom, Karta Kooner, Tomi Koivisto, David Kraljic, Sven Krippendorf, Andrei Linde,  Matteo Licheri, Oscar Loaiza-Brito, Ratul Mahanta, Mahbub Majumdar, M.C. David Marsh, Damian Mayorga Pena, Christoph Mayrhofer, Anupam Mazumdar,  Martin Mosny, Sebastian Moster, David F.~Mota, Francesco Muia, Maria Mylova, Peter Nilles, Sirui Ning, Gustavo Niz, Amin Nizami, Detlef Nolte, Carlos N\'u\~nez, Eimear O'Callaghan, Yessenia Olgu\'in-Trejo, Stefano Orani, Ogan Ozsoy, Sonia Paban, Antonio Padilla, Eran Palti, Francisco Pedro, Nicola Pedron, Veronica Pasquarella, Andrew Powell, Cari Powell, Mariano Quiros, Norma Quiroz, Ra\'ul Rabadan, Sa\'ul Ramos-S\'anchez, Govindan Rajesh, Seif Randjbar-Daemi, Filippo Revello, Soo-Jong Rey, Andreas Ringwald, Diederik Roest, Robert Rosati, Esteban Roulet, Markus Rummel, Alberto Salvio, Marco Scalisi,  Andreas Schachner, Jonas Schmidt, Matthias Schmitz, Marco Serra, Kajal Singh, Kuver Sinha, Raffaele Savelli, Ravi Sharma, Pramod Shukla, Aninda Sinha, Yoske Sumitomo, Kerim Suruliz, Gianmassimo Tasinato, Flavio Tonioni, Carlo Trugenberger, Angel Uranga, Gian Paolo Vacca, Vivian Poulin,  Roberto Valandro, Leo Van Nierop, Gonzalo Villa, Alexander Westphal, Matthew Williams, Danielle Wills, Lukas Witkowski, Timm Wrase, Marco Zagermann, Ren-Jie Zhang. 
FQ wants to particularly thank Cliff Burgess for 40 years of very enjoyable collaborations, some of them reported here.
 AM is supported in part by the SERB, DST, Government of India by the grant MTR/2019/000267.
 The work of FQ has been partially supported by STFC consolidated grants ST/P000681/1, ST/T000694/1. JC has been partially supported by STFC consolidated grants ST/P000770/1 and ST/T000864/1.
IZ is partially funded by STFC grant ST/T000813/1 and thanks {\em Fondazione Cassa di Risparmio}, Bologna for financial support during her visit to Bologna, where part of this work was developed.\\

For the purpose of open access,
the authors have applied a Creative Commons Attribution (CC BY) licence to any Author Accepted Manuscript version arising. Data access statement: no new data were generated for this work.

\enddocument 
 
 \newpage
 
 
 
 
 
 
 
 
 
 
 
 