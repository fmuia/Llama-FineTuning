\startdocument

\section{Introduction}



We are living in a golden age for cosmology. The exquisite precision with which the power spectrum of density perturbations of the cosmic microwave background (CMB) has been measured over the past 25 years is simply spectacular. The surprising discovery of the accelerated expansion of the universe in the present epoch, also around  25 years ago, has given rise to arguably the biggest puzzle in physics: dark energy. This, combined with ever improving observations of the large-scale structure of the universe and the compelling evidence for the existence of dark matter,  has made cosmology a precision science dominated by big data at all scales. The standard model of cosmology, $\Lambda$CDM, 
has only a handful of parameters but provides an accurate match to most observations. A successful scenario, inflation, has emerged as the standard description of early universe cosmology that addresses the main puzzles of the Big Bang model (flatness, horizon, monopole problems) and, most importantly, provides the seed for the density perturbations imprinted in the CMB. Its theoretical predictions fit remarkably well with observations.

However, unlike the Standard Model of particle physics, which is a well defined theory with concrete predictions, the standard model of cosmology, including inflation, is not based on an underlying theory. This is a fundamental  issue given  that early universe cosmology involves  temperatures and energy scales which are  higher than those probed in our  laboratories. At these scales, we do not have a complete theory. Furthermore, in contrast to particle physics, gravity cannot be neglected when addressing questions in cosmology. Therefore, in order to address the physics of the early universe, we need to have a well defined theory of gravity which also includes all other interactions. There are also puzzles at the longest wavelengths.
Formulating quantum mechanics in an accelerating spacetime (such as the present universe) is subtle as it is tied to various conceptual issues of quantum gravity.  Over more than 35 years, string theory has emerged as the most promising candidate for providing a consistent quantum 
 theory combining gravity with all other particles and interactions. Yet, string theory still lacks concrete predictions that can be tested experimentally with today's technology   and needs to be developed further so that it can be
confronted with potential observations in the not too distant future. Establishing the connections between string theory and cosmology is therefore one of primary importance for fundamental physics

Not surprisingly, there have been many attempts to extract information from string theory regarding its potential cosmological implications. This is not an easy task as our understanding of string theory is still incomplete. It is not yet possible to answer questions tied to the Big Bang singularity in the
context of string theory. Nevertheless, there are many cosmological questions that can be addressed by string theory. In particular, deriving models of cosmological inflation from string theory is a difficult but achievable task, as is the physics from the end of inflation to the present epoch which
covers a range of energies and temperatures many orders of magnitude and may, in a logarithmic scale, correspond to up to half of the expansion of the universe.  In addition, string theory can lead to various exotic phases in the (early) universe or consistency conditions that have  distinct implications for cosmology. These {\it alternatives} are the least understood but some of  the  most exciting directions to explore in string cosmology.

String cosmology is a natural meeting point for many disciplines. String theory has a large number of degrees of freedom in addition to those associated with the Standard Model and gravity. Many of these can be  light and of direct relevance for cosmology. Of particular importance are {\it moduli} -- the fields that control the shape and size of the extra dimensions, thereby setting the magnitude
 of couplings in the 4-dimensional effective field theory. Moduli are also ideal candidates for inflatons and often acquire non-trivial time dependence in post-inflationary string cosmology leading to differences from the standard cosmological timeline. In the present epoch, all moduli must be pinned to their minima or be very slowly rolling. Thus, whether it be the  early universe or the present epoch, understanding the potential energy functional for moduli fields and their dynamics is central to string cosmology. Therefore, string cosmology requires a deep  knowledge of string compactifications (dimensional reduction and  derivation of low energy effective actions that
 arise from string theory) -- this  involves  many aspects of modern  mathematics. The study of
 alternatives often requires  understanding string theory in novel regimes and has the potential to provide an  answer to the question:  What is string theory? Furthermore, addressing  questions such as the dimensionality of spacetime and, as most theorists believe, how spacetime itself may emerge from a fundamental theory, may lie in the domain of string cosmology. 
 
 Finally, from the point of view of a pragmatic cosmologist, string theory can be thought of as a black box which continually generates interesting models and scenarios. These have served as 
 a useful driver for both theory and observational targets in cosmology.  String cosmology thus brings together many areas and its study is not only central to our understanding of fundamental physics but also for advances in these areas.
 
This review aims to give a concise overview of the state of the art in  the subject. It is structured as  follows. Sec. 2 provides a brief review of
cosmology. After quickly going through Freedman-Lemaitre-Robertson-Walker (FLRW) cosmology and the history of the universe in the standard model of cosmology, we describe the physics of inflation. We discuss how inflation provides a theory for generating inhomogeneities in the universe, thereby allowing  it to connect to precision observational cosmology. We also introduce 
quintessence -- the possibility that the acceleration of the present universe is due to a slowly rolling scalar field. 

Sec. 3 deals with string compactifications and moduli fields. After providing a general overview of moduli, we discuss moduli stabilisation in various string theories. A summary of various scenarios to obtain de Sitter space (as a model of the universe in the present epoch) is provided, emphasising the general achievements and challenges. 

Sec. 4 is on inflation in string theory.  Here, we begin by describing why it is necessary to embed models of inflation
into theories of quantum gravity and the challenges for inflation in string theory. We present a list of well-established string theoretic models of inflation classified according to the form of their potential. We also give a ``report card" in the form of a table which shows how each of these models fares when confronted with observations. 

Sec. 5 is on the post-inflationary epoch between the end of inflation and the start of the Hot Big Bang. We start by discussing reheating in the context of string cosmology and identify the cosmological moduli problem, which is a generic outcome of string cosmology. We go on to describe modifications of the standard cosmological timeline (such as epochs of {\it moduli domination} and {\it kination}) which are natural in string cosmology. Opportunities and challenges in the context of dark matter and dark radiation are summarised and concrete sources of inhomogeneities and gravitational waves are identified, such as {\it oscillons}.

Sec. 6 is on dark energy (the present day constituent of the energy budget of the universe that is driving acceleration) in string theory. It is divided into two main parts -- dark energy arising from a cosmological constant term (a de Sitter solution) and dynamical dark energy (quintessence). In the
first part, after describing the cosmological constant problem we discuss how the {\it string landscape} (an enormously large number of string vacua with finely spaced values of the cosmological constant) offers a potential, if controversial, solution to the problem. In the second part, interesting avenues to construct models of quintessence in string theory are discussed and the associated challenges are outlined.

Sec. 7 deals with alternatives to the standard cosmology. We discuss string gas cosmology, the ekpyrotic/cyclic universe, rolling tachyon cosmology, pre-Big Bang cosmology, S-branes, holographic models and  models including creation or decay to nothing. This section also discusses  the {\it swampland} approach, which aims at determining consistency conditions (and their physical implications) that an effective field theory must satisfy so that it can be embedded in string theory or any theory of quantum gravity. We conclude in Sec. 8.

\enddocument

 \newpage
 
 
 
 
 
 
 
 
 
 
 
 
 
 