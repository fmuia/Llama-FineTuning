

\section{Conclusions and Outlook \label{sec:conclusions}}


A shift-symmetric coupling between the inflation sector and a (non-abelian) gauge sector opens up new avenues to probe the microphysics of inflation. Based on earlier works studying the coupling of the inflaton field to abelian~\cite{Barnaby:2010vf} and non-abelian~\cite{Adshead:2012kp,Dimastrogiovanni:2012st,Dimastrogiovanni:2012ew,Adshead:2013qp,Adshead:2013nka} gauge groups, we demonstrate here how the former can be understood as a limit of the latter in the far past. To this end, we extend and complement the analysis of CNI (see Refs.~\cite{Dimastrogiovanni:2012ew,Adshead:2013qp,Adshead:2013nka}) in the following ways: (i) Most importantly, we propose a mechanism for the dynamical generation of the homogeneous isotropic gauge field background, a key ingredient to CNI, starting from standard Bunch--Davies initial conditions in the infinite past. The evolution of the background is then governed by two competing effects: the classical background motion and the stochastic motion due to the strong growth of gauge field perturbations. We demonstrate how the system converges to an attractor solution, where the gauge field background is dominated by the classical motion. (ii) This attractor solution is more general than the solution employed in~\cite{Adshead:2012kp}, and in particular also allows us to study the parameter space in which the backreaction of the gauge fields on the equation of motion of the inflaton is small. (iii) We provide relatively simple explicit expressions for the equations of motion of the Fourier components of all physical perturbations modes, after integrating out all gauge degrees of freedom and constraint equations. Due to a slightly different choice of basis, our expressions are somewhat simpler than those found in the literature, in particular when studying only the gauge field fluctuations. (iv) Although the bulk of our analysis is performed for the linearized system of perturbations, we highlight the importance of non-linear contributions in the scalar sector, giving an estimate for the resulting contribution to the scalar power spectrum. Finally, we emphasize that the accuracy and efficiency of the proposed mechanism for the generation of the gauge field background should be verified in numerical lattice simulations. Research in this direction is ongoing (see e.g.\ Refs.~\cite{Adshead:2015pva,Cuissa:2018oiw} for recent developments) and we hope that this paper will arouse interest in these questions.

In the inherently non-abelian regime of our analysis, we reproduce the phenomenology of CNI: a single tensor (i.e.\ helicity 2) gauge field mode is tachyonically enhanced. A linear coupling of this mode to one helicity of the tensor metric perturbation sources a strongly enhanced, chiral gravitational wave background.  We point out that the instability observed in the scalar sector in Ref.~\cite{Dimastrogiovanni:2012ew} only occurs in a tiny part of the parameter space (achievable only for large gauge couplings when the transition to the non-abelian regime can occur for very low inflaton velocities close to the theoretical threshold value), and is in particular absent for the parameter point studied in this paper. Large contributions to the scalar power spectrum are however sourced by non-linear contributions. We propose a method to  estimate this non-linear contribution based on a generalization of the procedure described in~\cite{Linde:2012bt} and it should be seen as an order-of-magnitude estimate only. Interestingly, we have found that the non-linear corrections are expected to quickly dominate over vacuum and linear contributions inducing a strong enhancement in the scalar power spectrum at small scales. Such an enhancement may lead to a rich phenomenology (e.g.\ PBH formation~\cite{Linde:2012bt, Domcke:2017fix,Garcia-Bellido:2016dkw}, $\mu$-distortions~\cite{Meerburg:2012id}, etc).

The transition between an abelian limit at early times and a non-abelian regime at later times provides a natural way to obtain results in agreement with all CMB constraints while obtaining a phenomenologically very interesting enhanced scalar and tensor power spectra at small scales. The key parameter driving this transition is the instability parameter $\xi$ (see Eq.~\eqref{eq:rev_xi}), measuring the velocity of the inflaton. In single-field slow-roll inflation models, this parameter is small at early times but increases over the course of inflation. This parameter controls the tachyonic enhancement of the gauge field perturbations, as well as the amplitude of the homogeneous background solution. For the parameter point studied in this paper, the predicted enhancement of the scalar and tensor power spectrum is out of reach of current and upcoming experiments. A comprehensive exploration of the parameter space of these models, in particular regarding the prospects of detecting the gravitational wave signal with LIGO, LISA or the Einstein Telescope, is left for future work.

For the numerical study in this paper, we focused on a parameter point for which the gauge friction on the inflaton motion was small at the point of transition between the abelian and non-abelian regime. We leave a study of the opposite regime, corresponding to the usual so-called magnetic drift regime of CNI to future work. In this case, the dynamics at the transition point change more violently, requiring a more sophisticated modeling of this transition regime. In a similar spirit, the transition region itself deserves more attention. In this paper we simply refrained from making any statements very close to the transition region, to avoid sensitivity to the precise modeling procedure of this transition. Clearly, this calls for further improvement. Finally, the restriction to a linearized system of perturbations in the bulk of our analysis limits our ability to explore parameters for which the transition to the non-abelian regime occurs significantly before the end of inflation. This constraint is already problematic for the parameter point studied here, and likely becomes even more relevant in the parameter space most promising for direct gravitational wave searches, calling for a full non-linear treatment of the system. 

Our analysis is a further step towards the embedding of axion inflation models in a fully realistic particle physics description of the early Universe. In particular, our unified framework for abelian and non-abelian couplings may be applied to more complex gauge groups such as $\mathrm{SU}(2) \times \mathrm{U}(1)$ and eventually the full Standard Model of Particle Physics.




