In general all the single field models of inflation can be described by an Einstein frame Lagrangian that is similar to the one shown in Eq.~\eqref{non_minimal:einstein_frame} i.e.\ by:
\begin{equation}
	\label{attractors:einstein_frame}
	\mathcal{S} = \int \textrm{d}^4 x \sqrt{-g} \left[ \frac{R}{2} - K(\phi)X -V_E(\phi)  \right] \ .
\end{equation}
A particular case are the so-called \emph{T-models}~\cite{Kallosh:2013hoa,Kaiser:2013sna,Kallosh:2013daa,Kallosh:2013tua,Kallosh:2013yoa,Kallosh:2014rga,Kallosh:2015lwa} which lead to the class of $\alpha$-attractors. \emph{T-models} are described by the action of Eq.~\eqref{attractors:einstein_frame} with:
\begin{equation}
\label{eq:Tmodels}
	K(\phi) = \left(1 - \frac{\phi^2}{6 \alpha} \right)^{-2} \ , \qquad \qquad V_E(\phi) = \frac{m^2 \phi^2}{ 2} \ .
\end{equation}
Other interesting examples are the case of Higgs inflation introduce by Bezrukov and Shaposhnikov~\cite{Bezrukov:2007ep,Bezrukov:2009db} and its generalization. in terms of the so-called attractor at strong coupling~\cite{Kallosh:2013tua}: 
\begin{equation}
	\label{attractors:model_def}
	K(\phi) = \frac{1+\varsigma h(\phi) + 3/2 \, \varsigma^2 h_{,\phi}^2(\phi)}{(1+\varsigma h(\phi))^2} \ , \qquad \qquad V_E(\phi) = \lambda^4 \frac{ h^2(\phi)}{ \left( 1 + \varsigma h(\phi) \right)^2} \ ,
\end{equation}
where $\lambda$ is a mass scale (that fixes the normalization of the inflationary potential) and $h(\phi)$ is a generic function\footnote{Higgs inflation corresponds to $h(\phi)=\phi^2$.} of $\phi$.



Over the last years these models have received a lot of attention because they predict values of $n_s$ and $r$ that are in good agreement with the values that are favored by Planck data~\cite{Ade:2015xua,Ade:2015lrj}. In the case of T-models we have:
\begin{equation}
	n_s -1 = - \frac{2}{N} \ , \qquad \qquad r = \frac{12 \alpha}{N^2} \ ,
\end{equation}
and in the case of the attractor at strong coupling (i.e.~$\varsigma \gg 1$):
\begin{equation}
	n_s -1 = - \frac{2}{N} \ , \qquad \qquad r = \frac{12 }{N^2} \ .
\end{equation}
We can immediately notice that, for both these models:\footnote{However, it is crucial to stress that these expressions for $n_s -1$ and $r$ only hold for large values of $N$. This will be particularly relevant for the possibility of observing the GW spectrum generated by these models.}
\begin{equation}
\epsilon_H \simeq \frac{\mathcal{O}(1)}{N^2} \ .
\end{equation}
Interestingly, using the classification of inflation models~\cite{Mukhanov:2013tua,Roest:2013fha,Binetruy:2014zya} based on first slow-roll parameter $\epsilon_H$:
\begin{equation}
	\epsilon_H \simeq \frac{\beta_p}{N^p} + \mathcal{O}(1/N^{p+1}) \ ,
\end{equation}
all these models fall in the same universality class (i.e.\ the Starobinsky-like class\footnote{The wording `Starobinsky-like' refers here to a class of models characterized by a scalar potential which asymptotes to an exponentially flat plateau. The original Starobinsky model~\cite{Starobinsky:1980te} is constructed from the Ricci-scalar which is not a pseudoscalar quantity.} with $p=2$). As shown in~\cite{Domcke:2016bkh}, in the context of higher-dimensional couplings with Abelian gauge fields this happens to be the most promising scenario from the point of view of possible observational signatures. As a consequence it seems interesting apply the framework developed in Sec.~\ref{sec:non-minimal} to this class of models.



In order to study the effect of non-minimal couplings on the dynamics and on the scalar and tensor power spectra, we focus on the family of models defined by Eq.~\eqref{attractors:model_def}. As a first example, let us consider models with $h(\phi) = \phi$ so that:
\begin{equation}
	\label{attractors:model_example}
	K(\phi) = \frac{1+\varsigma \phi + 3/2 \, \varsigma^2 }{(1+\varsigma \phi)^2} \ , \qquad \qquad V_E(\phi) = \lambda^4 \frac{ \phi^2}{ \left( 1 + \varsigma \phi \right)^2} \ .
\end{equation}
Notice that for $\varsigma = 0 $ this model simply reduces to the case of chaotic inflation with $V(\phi) = \frac{1}{2} m^2 \phi^2$ i.e.\ it corresponds to the $p=1$ class of~\cite{Domcke:2016bkh}. 
By increasing the value of $\varsigma$ we progressively flatten the potential, see left panel of Fig.~\ref{fig:potentials_and_xi}.\footnote{For all the models shown in this plot the constant $\lambda$ is fixed in order to respect the observed normalization of the scalar spectrum (according to~\cite{Ade:2015lrj}) at CMB scales. In particular this can be noticed by looking at the plot of Fig.~\ref{fig:scalar}, where we show the scalar spectra associated with all the different models discussed in this section.} This clearly induces a strong modification on the evolution of the inflaton field. In particular, the field excursion during inflation is decreased as $\varsigma$ increases and the evolution of $\xi$ is altered compared to the minimally coupled case (see right panel of Fig.~\ref{fig:potentials_and_xi}).
 Notice that for increasing values of $\varsigma$ the growth of $\xi$ (as a function of $N$) becomes slower. In particular the suppression of $\xi$ over a wide range of e-folds leads to a strong suppression in both the scalar and tensor power spectra.

\begin{figure}
\centering
\includegraphics[width=0.555\textwidth]{potentials}
\includegraphics[width=0.435\textwidth]{xi}
\caption{\textbf{Left panel:} Profile of the inflationary potential~\eqref{attractors:model_example} normalized to its value at the end of inflation, $V(\phi_\textrm{f})$, as a function of the field excursion $\phi - \phi_\textrm{f}$ for different values of the non-minimal coupling $\varsigma$ to gravity. Dots are used to denote the points that corresponds to $N = 60$ in the case with $\alpha/\Lambda = 0$. \textbf{Right panel:} Evolution of the parameter $\xi$ governing the effects of the gauge fields as a function of the number of e-folds $N$ (right plot).}
\label{fig:potentials_and_xi}
\end{figure}

 
As described in Sec.~\ref{sec:non-minimal}, the changes in the tensor spectrum are simply due to the different evolution of $\xi$. In particular, as for increasing values of $\varsigma$ the growth of $\xi$ becomes slower over a broad range of $N$-values, we expect the GW spectrum (which is exponentially sensitive to $\xi$) to be strongly suppressed. This effect is clearly visible in Fig.~\ref{fig:GW} where we compare the GW spectra for the models of Fig~\ref{fig:potentials_and_xi} with the sensitivity curves of some current (solid lines) and upcoming (dashed lines) direct GW detectors. For the millisecond pulsar timing arrays covering frequencies around $10^{-10}$~Hz, we show the constraint depicted in Ref.~\cite{Smith:2005mm}, the update from EPTA~\cite{vanHaasteren:2011ni} and the expected sensitivity of SKA~\cite{Kramer:2004rwa}. The laser interferometer space antenna (LISA)~\cite{LISA} is a future mission designed to probe the mHz range~\cite{Petiteau, Caprini:2015zlo,Bartolo:2016ami}, where as ground-based detectors are sensitive at a few 10 Hz (LIGO/VIRGO~\cite{TheLIGOScientific:2016wyq}). It is interesting to notice that for $\varsigma \sim 1$ the spectrum is marginally detectable (for $\mathcal{N}=10$) by LISA but not by advanced LIGO. For larger values of $\varsigma$ the suppression of the spectrum is so strong that the signal is well below the expected sensitivities of upcoming GW detectors.


\begin{figure}
\centering
\includegraphics[width=0.9\textwidth]{Production_GW}
\caption{Power spectrum of tensor perturbations for all the models of Fig.~\ref{fig:potentials_and_xi} (dashed lines corresponds to models with $\mathcal{N}=10$). We are also showing the sensitivity curves for (from left to right): milli-second pulsar timing, LISA, advanced LIGO. Current bounds are denoted by solid lines, expected sensitivities of upcoming experiments by dashed lines. See main text for details on these curves.}
\label{fig:GW}
\end{figure}

In Fig.~\ref{fig:scalar} we show the scalar power spectrum (corresponding to Eq.~\eqref{non_minimal:scalar_final}) for all the models shown in the previous plots. Notice that, in order to respect the COBE normalization (that sets the amplitude of the spectrum  at CMB scales) all the spectra meet at the same value for $N\simeq 60$. Moreover, it is worth mentioning that all the models shown in this section are in agreement\footnote{Except for the model with $\varsigma = 0.01$, for which the predicted value of $r$ is marginally excluded by Planck~\cite{Ade:2015lrj} at $95\%$ CL.} with the constraints set by Planck on $n_s$ and $r$ and on the generation of primordial non-Gaussianities.\footnote{ This can directly be checked from Fig.~\ref{fig:potentials_and_xi}, $\xi|_\text{CMB} \lesssim 2.5$.} Interestingly, for none of the models shown in this plot the spectrum features a peak. This can be explained by considering the expression of $K(\phi)$ in terms of $N$. While for $\varsigma \ll 1 $ $K(\phi)$ is essentially constant, for sizable values of $\varsigma$ we have $K(\phi) \propto 1/(\varsigma N)^2$ to leading order in $N$. By numerically solving the evolution it is possible to show that $K(\phi)$ becomes large only for $N\simeq 2 \div 4 $ depending on the model, and the suppression of $\xi$ is not strong enough to disrupt its monotonic growth before the end of inflation. As a consequence, in order for the spectrum to feature a peak, we require a strong growth of $K(\phi)$ at sufficiently large values of $N$. Such a behavior can be achieved by considering different parameterizations for $h(\phi)$.


\begin{figure}
\centering
\includegraphics[width=0.9\textwidth]{Production_scalar}
\caption{Scalar power spectrum for all the models of Fig.~\ref{fig:potentials_and_xi} (dashed lines corresponds to models with $\mathcal{N}=10$).
 The lower horizontal line denotes the observed normalization at CMB scales (according to~\cite{Ade:2015lrj}) and the upper horizontal line is the estimate of~\cite{Linde:2012bt} for the PBH bound. More details on this bound on the generation of PBHs are reported in Sec.~\ref{sec:PBHs}.}
\label{fig:scalar}
\end{figure}



As discussed in Ref.~\cite{Domcke:2016bkh}, the scalar and tensor spectra in axion inflation can be well understood in terms of universality classes of inflation~\cite{Mukhanov:2013tua,Roest:2013fha,Binetruy:2014zya}, categorized by the scaling of $\epsilon_H(N)$ in the regime where the back-reaction of the gauge fields is weak. This is immediately understood (for a minimal coupling to gravity) since the growth of both the scalar and tensor spectrum is driven by $\xi \propto \sqrt{\epsilon_H}$. However, when a non-minimal coupling is considered, the instability is controlled by $\xi \propto \sqrt{\epsilon_H/K}$ (see Eq.~\eqref{non_minimal:epsilon_H}). Whereas the examples depicted so far correspond to $\epsilon_H/K \propto 1/N^p$ with $p \lesssim 1$~\cite{Binetruy:2014zya},\footnote{While it is easy to see that $p = 1$ for $\varsigma = 0$ (see~\cite{Binetruy:2014zya}), this statement may seem surprising for $\varsigma \neq 1$, when the scalar potential asymptotically approaches the exponentially flat Starobinsky-like potential (that for a minimal coupling is characterized by $\epsilon_H \propto 1/N^2$~\cite{Binetruy:2014zya}). This subtlety arises due to the coupling of the gauge fields to the canonically normalized inflaton field of the Jordan frame. By considering the expression of $K(\phi)$ given in Eq.~\eqref{attractors:model_def} and using the definition $N \equiv \int \sqrt{K/2\epsilon_H} \, \textrm{d}\phi$, one finds that $\xi \propto \sqrt{\epsilon_H/K}$ is in fact independent of $N$ at small values of $N$ (ignoring here a possible black-reaction of the gauge-fields).} 
different behaviors can be achieved. In particular here, in order to obtain a scalar spectrum which rises at larger values of $N$ (i.e.\ at an earlier stage during inflation) we require a quicker increase of  $\epsilon_H(N)$, with decreasing $N$. At the same time, in order to obtain a suppression of $\xi$ which leads to the appearance of a bump in the spectra, we also require the kinetic function $K(N)$ to become sizable at sufficiently large values of $N$. A simple example of this type can obtained by choosing:
\begin{equation}
	\label{attractors:inv_definition}
 	h(\phi) = (1 - 1/\phi) \ , \qquad \qquad  V_E(\phi) = \lambda^4 \frac{ (\phi-1)^2}{ \left( \phi + \varsigma (\phi-1) \right)^2} \ ,
 \end{equation} 
 which for a vanishing non-minimal coupling and a vanishing coupling to gauge fields ($\varsigma = 0$, $\alpha = 0$) leads to $\epsilon_H \propto 1/N^{4/3}$ while the kinetic function $K(N)$ interpolates from an approximately constant behaviorw at large $N$ to $K(N) \propto e^{-N}$ at small $N$. We will also return to models of this type when discussing PBH dark matter.

\begin{figure}[t]
%
\centering
\includegraphics[width=0.9\textwidth]{xiinv.pdf}

\caption{Evolution of $\xi$ as a function of $N$ for models based on Eq.~\eqref{attractors:inv_definition}. The model parameters are chosen to obey CMB and PBH constraints, see main text for details. In particular,  the depicted curves are obtained for couplings to the gauge-fields of $\alpha/\Lambda \simeq \{ 97, 63, 71, 74 \}$ (for increasing value of the non-minimal coupling to gravity $\varsigma$, respectively).}
\label{fig:xi_inv}
%
\end{figure}

In Fig.~\ref{fig:xi_inv} we depict the evolution of the parameter $\xi$  for some representative models of the class defined by Eq.~\eqref{attractors:inv_definition}. As before, the normalization of the scalar potential is fixed by the COBE normalization that fixes the amplitude of the scalar power spectrum at CMB scales and we ensure agreement with all CMB constraints. The remaining parameters are chosen to obtain sizable gauge-field induced effects while conveying an idea of the possible parameter space. 
 We clearly see the peaked features around $N \simeq 20 - 40$ in all spectra with a significant value of the non-minimal coupling $\varsigma$, which translates into a broad peak in the scalar and tensor spectra, depicted in Fig.~\ref{fig:spectra}.
For increasing values of $\alpha/\Lambda$ and $\varsigma$ the bump in the spectra shifts to larger values of $N$ i.e.\ to larger scales. This is consistent with the physical intuition: For larger values of $\alpha/\Lambda$ the increase in the spectrum arises earlier (see also~\cite{Domcke:2016bkh}), for larger values of $\varsigma$ the subsequent suppression due to the non-canonically normalized kinetic term takes effect at earlier times.
Moreover, increasing $\varsigma$ we reduce the field excursion during inflation, leading to a suppression of $\xi$ over a broad range of $N$-values. As a consequence (and indeed consistently with Fig.~\ref{fig:GW} and Fig.~\ref{fig:scalar}), increasing the non-minimal coupling $\varsigma$ we note a suppression both in the scalar and tensor power spectra. For $\varsigma \simeq 63$ and $71$ the signal is below the predicted sensitivity of LISA by roughly an order of magnitude. Finally, we point out that the peaks in the scalar spectrum of Fig.~\ref{fig:spectra} correspond to values of $\xi \lesssim 4.6$. Contrary to the case of minimal coupling, we do hence not enter into the regime $\xi \gtrsim 4.8$ where perturbativity breaks down~\cite{Peloso:2016gqs} and where significant uncertainties become attached to the predictions for the scalar spectrum.


\begin{figure}[htb!]
%
\centering
\includegraphics[width=0.97\textwidth]{GWinverse.pdf}\vspace{5mm}
\includegraphics[width=0.97\textwidth]{Productioninverse.pdf}
%
\caption{Scalar and tensor spectrum for models based on Eq.~\eqref{attractors:inv_definition} with model parameters and color-coding as in Fig.~\ref{fig:xi_inv}.
 The experimental constraints (in grey) are as in Fig.~\ref{fig:GW} and Fig.~\ref{fig:scalar}, respectively.}
\label{fig:spectra}
%
\end{figure}



Let us conclude this section by summarizing the key feature of $K(\phi)$ and $V(\phi)$ which allow for the appearance of a peak in the spectra within the last 60 e-folds. A first requirement is that the instability occurs at sufficiently large values of $N$. This can be rephrased as the requirement that $\xi \propto \sqrt{\epsilon_H/K}$ grows sufficiently fast with decreasing $N$ for large values of $N$. After this first regime, we need the effects of the non-canonical kinetic term to become important. Concretely, in order to observe a bump in the spectra we require a rapid increase in $K(\phi)$ at a sufficiently large value of $N$ (for example at $N \simeq 20$). As the growth of $K(\phi)$ turns into a suppression of $\xi$, this mechanism shuts off the instability, leading to the appearance of a bump both in the scalar and tensor power spectra. To illustrate this we chose to study two example models as defined by Eq.~\eqref{attractors:model_def} and Eq.~\eqref{attractors:inv_definition}. However, one can also perform a more model-independent analysis. In particular, we can make predictions for the scalar and tensor spectra by directly parameterizing $K$ and $\epsilon_H$ as functions of $N$:
\begin{equation}
 	\epsilon_H \simeq \frac{\beta}{(1+N)^\alpha} \ , \qquad K \simeq 1 + \frac{\gamma}{N^\delta} \ .
 \end{equation} 
Note that one can always reconstruct the expressions for $V_E(\phi)$ and $K(\phi)$ by expressing $N$ as a function of $\phi$. Hence, in case of a detection of peaks in the scalar and tensor spectra one could determine the corresponding expressions for $V_E(\phi)$ and $K(\phi)$.