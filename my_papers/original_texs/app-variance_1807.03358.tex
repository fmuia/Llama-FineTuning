\section{ Computation of the variance of \texorpdfstring{$F \tilde F$}{FFdual} \label{app:variance_computation}}
In this appendix we are interested in computing the variation of $\langle F \tilde F \rangle$ with respect to its background value, which we employ to estimate the non-linear contribution to the scalar power spectrum in Sec.~\ref{subsec:PS}. In general this can be expressed as
\begin{equation}
 \delta (\langle F \tilde F \rangle ) = [F \tilde F - \langle F \tilde F \rangle ]_{\delta \phi = 0} + \frac{\partial \langle F \tilde F \rangle }{\partial \dot \phi} \dot{\delta \phi} \equiv \delta_{\vec{EB}} + \frac{\partial \langle F \tilde F \rangle }{\partial \dot \phi} \dot{\delta \phi} \,,
\end{equation}
where we are using the sign convention $\dot \phi > 0$, $\langle F \tilde F \rangle >  0$ (see \cite{Linde:2012bt}). Notice that this corresponds to considering the variation of $\langle F \tilde F \rangle$ due to variations of $\delta \phi$ (second term) plus the variations of $\langle F \tilde F \rangle$ due to the variations of the gauge fields. In order to perform the computation of the latter it is useful to introduce the electric and magnetic fields (in conformal time)
\begin{equation}
  F^{a}_{0i} \equiv - a^2(\tau) E_{i}^a \, , \qquad \qquad F^{a}_{ij} \equiv \varepsilon^{ijk} a^2(\tau) B_{k}^a \, .
\end{equation}
With these definitions we can trivially show that:
\begin{eqnarray}
  \label{eq:FF_EB}
  F^a_{\mu \nu} F^a_{\rho \sigma} g^{\mu \rho} g^{\nu \sigma } & = & - \frac{2}{a^4} F^a_{0 i} F^a_{0 i} + \frac{1}{a^4} F^a_{i j} F^a_{i j} = - 2 \left[ \left( \vec{E}^a \right)^2 - \left( \vec{B}^a \right)^2 \right] \, . \\
  \label{eq:FFdual_EB}
  F^a_{\mu \nu} \tilde{F}^a_{\rho \sigma} & = & \frac{4}{2 \sqrt{-g}} F^a_{0 i} F^a_{j k} \varepsilon^{ijk} = - 2 E^a_{i} B_{l}^a \varepsilon^{ijk}  \varepsilon^{ljk} = - 4 \vec{E}^a \cdot \vec{B}^a \, .
\end{eqnarray}
We can then proceed by computing the expressions of $E^a_{i}$ and $B^a_{i}$ (neglecting terms depending on $A_0^a$) up to second order:
\begin{eqnarray}
  E^a_{i} & = & - \frac{1}{a^2} \left[\delta^a_i \partial_0 f + \partial_0 \delta A^a_i  \right]\,, \\
  B^a_{i} & = & \frac{1}{ a^2} \left[ \varepsilon_{ijk} \partial_j \delta A^a_k + e f^2 \delta^a_i  - e f \delta A^i_a  +e \frac{\varepsilon_{ijk} \varepsilon^{abc}}{2} \delta A^b_j  \delta A^c_k  \right]\, , 
\end{eqnarray}
where we have neglected terms proportional to the helicity 0 quantity $\delta A^a_a$ since they do not feature the exponential enhancement present in the $e_{+2}$ mode. Notice that setting $e = 0$ we can easily recover the usual abelian terms.\\


\noindent With this notation it is now trivial to check that the expectation value of $F \tilde{F}$ can be expressed as:
\begin{equation}
  \langle F^a_{\mu \nu} \tilde{F}^{a \, \mu \nu} \rangle = - 4 \langle \vec{E}^a \cdot \vec{B}^{a } \rangle \equiv  - 4 T_0 \, ,
\end{equation}
and the variance of $F \tilde{F}$ can be expressed as
\begin{equation}
  \delta_{\vec{EB}}^2 = \langle ( F^a \tilde{F}^a )^2 \rangle  -  \langle  F^a \tilde{F}^a  \rangle ^2 = 16\left[  \langle  E_i^a E_j^b \rangle \langle B_i^a B_j^b \rangle + \langle E_i^a B_j^b \rangle \langle B_i^a E_j^b \rangle \right] \, ,
\end{equation}
which has exactly the same shape as in the abelian limit (for comparison see appendix A of~\cite{Linde:2012bt}). At this point it is useful to introduce 
\begin{eqnarray}
  T_1  \equiv \left \langle E^a_{i}  E^b_{j} \right \rangle \left \langle B^a_{i}  B^b_{j} \right \rangle \, , \qquad \qquad T_2  \equiv \left \langle E^a_{i} B^b_{j} \right \rangle \left \langle B^a_{i}  E^b_{j} \right \rangle \, ,
\end{eqnarray}
so that the two contributions can be computed independently. Before substituting the explicit expressions of $E$ and $B$ it is important to notice that (i) in order to compute the variance we only need terms that are exactly quadratic in the fluctuations and (ii)  the base vectors of the $\pm 2$ helicity modes are traceless in all bases (implying $\delta^a_i \left[e_{\pm2} (\vec{k})\right]^a_i = 0 $ for all $\vec{k}$). 
% 
We can now  directly compute 
\begin{align}
  \left \langle E^a_{i}  E^b_{j} \right \rangle  =& \frac{1}{a^4} \left \langle \partial_0 \delta A^a_i \partial_0 \delta A^b_j \right \rangle \, , \\
  \left \langle B^a_{i}  B^b_{j} \right \rangle =& \frac{1}{a^4} \left \langle  \varepsilon_{ilk} \varepsilon_{jnm} \partial_l   \delta A^a_k  \partial_n \delta A^b_m + e^2 f^2 \delta A^i_a \delta A^j_b  - e f \varepsilon_{ilk} \partial_l \delta A^k_a \delta A^b_j  - e f \varepsilon_{jnm}  \delta A^i_a \partial_n \delta A^b_m \right \rangle \nonumber \\
  \equiv&  \left \langle B^a_{i}  B^b_{j} \right \rangle_1 +  \left \langle B^a_{i}  B^b_{j} \right \rangle_2 +  \left \langle B^a_{i}  B^b_{j} \right \rangle_3 +  \left \langle B^a_{i}  B^b_{j} \right \rangle_4 , \\
  \left \langle E^a_{i} B^b_{j} \right \rangle =& - \frac{1}{a^4} \left \langle \partial_0 \delta A^a_i \varepsilon_{jnm} \partial_n \delta A^b_m  - e f  \partial_0 \delta A^a_i \delta A^j_b +  e \delta^a_i \partial_0 f \frac{\varepsilon_{jnm} \varepsilon^{bdh}}{2} \delta A^d_n  \delta A^h_m \right \rangle \nonumber \\
  \equiv&  \left \langle E^a_{i}  B^b_{j} \right \rangle_1 +  \left \langle E^a_{i}  B^b_{j} \right \rangle_2 +  \left \langle E^a_{i}  B^b_{j} \right \rangle_3 \, .
\end{align}
Notice that again setting $e = 0$ our expressions reduce to the abelian case. 

At this point we can expand $ \delta A^a_\nu(\tau, \vec x)$ in terms of its Fourier modes as in Eq.~\eqref{eq:Fourier_A}. Notice that in general the basis vectors satisfy 
\begin{equation}
   \left[ e_{\lambda, \nu}^a (\hat{k}) \right]^* = e_{\lambda, \nu}^a (-\hat{k}) \, , \qquad \qquad i \varepsilon_{ijl} k_j e_{\lambda, l}^a (\hat{k}) = \text{sgn}(\lambda) |\vec{k}| e_{\lambda, i}^a (\hat{k}) \, .
\end{equation}
Moreover, the helicity $\pm2$ vectors are symmetric, \emph{i.e.,} $e_{\pm2, i}^a = e_{\pm2, a}^i$. Using the properties of the basis vectors it is possible to show that (from now on we restrict our analysis to the $+2$ mode only)
\begin{equation}
  \left \langle E^a_{i}  E^b_{j} \right \rangle  = \frac{1}{a^4} \int \frac{\textrm{d} k \textrm{d} \Omega_{\vec{k}} }{(2 \pi)^{3}} \, k^2   \frac{ \partial_0 \tilde{\delta A}_{+2}(\tau,  k) \left[ \partial_0 \tilde{\delta A}_{+2}(\tau,  k) \right]^*  e_{+2, i}^a (\hat{k}) \left[ e_{+2, j}^b (\hat{k}) \right]^*  +h. c.}{2}  \, ,
\end{equation}
and analogously for the terms of $ \left \langle E^a_{i} B^b_{j} \right \rangle$ and $ \left \langle B^a_{i} B^b_{j} \right \rangle$
\begin{align}
  2 \left \langle E^a_{i} B^b_{j} \right \rangle_1  = & - \frac{1}{a^4} \int \frac{\textrm{d} k \textrm{d} \Omega_{\vec{k}} }{(2 \pi)^{3}} \, k^3 \left\{ \partial_0 \tilde{\delta A}_{+2}(\tau,  k) \left[ \tilde{\delta A}_{+2}(\tau,  k) \right]^*  e_{+2, i}^a (\hat{k}) \left[ e_{+2, j}^b (\hat{k}) \right]^*  +h. c. \right \} \, , \\
  2 \left \langle E^a_{i} B^b_{j} \right \rangle_2  = & \frac{ e f }{a^4} \int \frac{\textrm{d} k \textrm{d} \Omega_{\vec{k}} }{(2 \pi)^{3}} \, k^2 \left\{ \partial_0 \tilde{\delta A}_{+2}(\tau,  k) \left[ \tilde{\delta A}_{+2}(\tau,  k) \right]^*  e_{+2, i}^a (\hat{k}) \left[ e_{+2, j}^b (\hat{k}) \right]^*  +h. c. \right \} \, ,\\
  2 \left \langle E^a_{i} B^b_{j} \right \rangle_3  = & - \frac{e \partial_0 f}{a^4} \int \frac{\textrm{d} k \textrm{d} \Omega_{\vec{k}} }{(2 \pi)^{3}} \, k^2 \left\{ \tilde{\delta A}_{+2}(\tau,  k) \left[ \tilde{\delta A}_{+2}(\tau,  k) \right]^*  
  \times \right. \nonumber   \\  & \left. 
  \delta^a_i \frac{\varepsilon_{jnm} \varepsilon^{bdh}}{2} \left[ e_{+2, n}^d (\hat{k}) \right] \left[ e_{+2, m}^h (\hat{k}) \right]^*  +h. c. \right \} \, , \\
  2 \left \langle B^a_{i}  B^b_{j} \right \rangle_1  = & \frac{1}{a^4} \int \frac{\textrm{d} k \textrm{d} \Omega_{\vec{k}} }{(2 \pi)^{3}} \, k^4 \left\{ \tilde{\delta A}_{+2}(\tau,  k) \left[ \tilde{\delta A}_{+2}(\tau,  k) \right]^*  e_{+2, i}^a (\hat{k}) \left[ e_{+2, j}^b (\hat{k}) \right]^*  +h. c. \right \} \, , \\
   2 \left \langle B^a_{i}  B^b_{j} \right \rangle_2  = & \frac{e^2 f^2}{a^4} \int \frac{\textrm{d} k \textrm{d} \Omega_{\vec{k}} }{(2 \pi)^{3}} \, k^2 \left\{ \tilde{\delta A}_{+2}(\tau,  k) \left[ \tilde{\delta A}_{+2}(\tau,  k) \right]^*  e_{+2, i}^a (\hat{k}) \left[ e_{+2, j}^b (\hat{k}) \right]^*  +h. c. \right \} \, , \\
   2 \left \langle B^a_{i}  B^b_{j} \right \rangle_3  = & \frac{-ef}{a^4} \int \frac{\textrm{d} k \textrm{d} \Omega_{\vec{k}} }{(2 \pi)^{3}} \, k^3 \left\{ \tilde{\delta A}_{+2}(\tau,  k) \left[ \tilde{\delta A}_{+2}(\tau,  k) \right]^*   e_{+2, i}^a (\hat{k}) \left[ e_{+2, j}^b (\hat{k}) \right]^*  +h. c. \right \}  \, ,\\
  2 \left \langle B^a_{i}  B^b_{j} \right \rangle_4  = & \frac{-ef}{a^4} \int \frac{\textrm{d} k \textrm{d} \Omega_{\vec{k}} }{(2 \pi)^{3}} \, k^3 \left\{ \tilde{\delta A}_{+2}(\tau,  k) \left[ \tilde{\delta A}_{+2}(\tau,  k) \right]^*  e_{+2, i}^a (\hat{k}) \left[ e_{+2, j}^b (\hat{k}) \right]^*  +h. c. \right \} \, .
\end{align}
Since $T_1 $ and $T_2 $ are respectively given by 
\begin{eqnarray}
  T_1 &=& \left \langle E^a_{i} E^b_{j} \right \rangle \left(  \left \langle B^a_{i} B^b_{j} \right \rangle_1 +\left \langle B^a_{i} B^b_{j} \right \rangle_2 +\left \langle B^a_{i} B^b_{j} \right \rangle_3 +\left \langle B^a_{i} B^b_{j} \right \rangle_4 \right) \, , \\ 
  T_2 &=& \left( \left \langle E^a_{i} B^b_{j} \right \rangle_1 +\left \langle E^a_{i} B^b_{j} \right \rangle_2 +\left \langle E^a_{i} B^b_{j} \right \rangle_3 \right) \left( \left \langle B^a_{i} E^b_{j} \right \rangle_1 +\left \langle B^a_{i} E^b_{j} \right \rangle_2 +\left \langle B^a_{i} E^b_{j} \right \rangle_3 \right) \, , \hspace{1cm}
\end{eqnarray}
we can immediatly see that, all the angular integrals reduce to three combinations:
\begin{eqnarray}
  \Omega_1 &\equiv& \int \textrm{d} \Omega_{\vec{k}} \ \textrm{d} \Omega_{\vec{q}} \ e_{+2, i}^a (\hat{k}) \left[ e_{+2, j}^b (\hat{k}) \right]^*  e_{+2, i}^a (\hat{q}) \left[ e_{+2, j}^b (\hat{q}) \right]^* \, , \\
  \Omega_2 &\equiv& \frac{3}{4} \int \textrm{d} \Omega_{\vec{k}} \ \textrm{d} \Omega_{\vec{q}} \ \varepsilon_{jnm} \varepsilon^{bdh} \left[ e_{+2, n}^d (\hat{k}) \right] \left[ e_{+2, m}^h (\hat{k}) \right]^* \varepsilon_{jlu} \varepsilon^{bpr} \left[ e_{+2, l}^p (\hat{q}) \right] \left[ e_{+2, u}^r (\hat{q}) \right]^* , \hspace{1cm} \\
  \Omega_3 &\equiv& \int \textrm{d} \Omega_{\vec{k}} \ \textrm{d} \Omega_{\vec{q}} \ e_{+2, i}^a (\hat{k}) \left[ e_{+2, j}^b (\hat{k}) \right]^* \delta^a_i \frac{\varepsilon_{jnm} \varepsilon^{bdh}}{2} \left[ e_{+2, n}^d (\hat{k}) \right] \left[ e_{+2, m}^h (\hat{k}) \right]^* \, ,
\end{eqnarray}
whose direct evaluation gives
\begin{equation}
   \Omega_1 = \frac{(4\pi)^2}{5} \, ,\qquad \qquad \Omega_2 = \frac{4\pi^2}{3} \, ,\qquad \qquad \Omega_3 = 0 \, . 
\end{equation}
Notice that $\Omega_3 = 0$ can be used to simplify $T_2$ as:
\begin{equation}
  T_2 = \left( \left \langle E^a_{i} B^b_{j} \right \rangle_1 +\left \langle E^a_{i} B^b_{j} \right \rangle_2  \right) \left( \left \langle B^a_{i} E^b_{j} \right \rangle_1 +\left \langle B^a_{i} E^b_{j} \right \rangle_2 \right) +\left \langle E^a_{i} B^b_{j} \right \rangle_3 \left \langle B^a_{i} E^b_{j} \right \rangle_3 \, ,
\end{equation}
At this point we are finally left with only integrals over the absolute value of the momenta. In order to perform these integrals, we will employ Eq.~\eqref{eq:NonabelianAsymptotics2}, defining $\tilde w(x)$ as follows:
\begin{align}
  w_{+2}(x) & \simeq e^{(\kappa-\mu)\pi}  \sqrt{4\pi} \left(\frac{\zeta(x)}{V(x)}\right)^{1/4}\mathrm{Ai}\left(\zeta(x)\right) \\
 &  \equiv e^{(\kappa-\mu)\pi} \tilde w(x) \,.
\end{align}
We have verified that integrating this approximate expression over the range $0 \leq x \leq 2 \xi$ agrees with the integral over the exact expression~\eqref{eq:AnalyticalSolution} extremely well, and we are moreover insensitive to the choice of the UV-cutoff. Let us start by computing for example:
\begin{equation}
\begin{aligned}
  4 \left \langle E^a_{i} E^b_{j} \right \rangle \left \langle B^a_{i} B^b_{j} \right \rangle_1   = \frac{\Omega_1 e^{4 \pi (\kappa - \mu)}}{(2 \pi)^6 a^8} \int  \frac{ \textrm{d} k  \, k^2 }{k} \left\{ \partial_0 \tilde w( x_k)  \partial_0 \tilde w( x_k)  \right \}      \int  \frac{\textrm{d} q  \, q^4}{ q} \left\{ \tilde w( x_q) \tilde w( x_q) \right \} \,,
\end{aligned}
\end{equation}
where we use the notation $x_k = - k \tau$ to keep track of the different momentum variables. We can then use $\partial_0 \tilde w( x_k)  = - k \tilde w( x_k)$ ($\prime$ here is used to denote the derivative with respect to $x_k$), $k = - x_k/ \tau$ and $\tau = - 1/(a H)$ to get
\begin{equation}
\begin{aligned}
  4 \left \langle E^a_{i} E^b_{j} \right \rangle \left \langle B^a_{i} B^b_{j} \right \rangle_1   = \frac{\Omega_1 e^{4 \pi (\kappa - \mu)}}{(2 \pi)^6} H^8 \int  \textrm{d} x_k  \, x_k^3  \left\{\tilde w'( x_k)  \tilde w'( x_k)   \right \}     \int \textrm{d} x_q  \, x_q^3 \ \left\{\tilde w( x_q)  \tilde w( x_q)   \right \}  \, .
\end{aligned}
\end{equation}
Since all the other terms can also be expressed as a product of two integrals it is useful to introduce the six integrals:
\begin{align}
  \mathcal{I}_1 & =  \int  \textrm{d} x  \, x^3 \  \tilde w'( x)  \tilde w'( x) \,, \qquad && 
  \mathcal{I}_2  =  \int  \textrm{d} x  \, x^3 \ \tilde w( x) \tilde w( x)  \,, \qquad 
  \mathcal{I}_3 & =  \int  \textrm{d} x  \, x \ \tilde w( x) \tilde w( x)  \,, \\ 
  \mathcal{I}_4 &  =  \int  \textrm{d} x  \, x^2 \ \tilde w( x) \tilde w( x) \,, \qquad &&
  \mathcal{I}_5  =  \int  \textrm{d} x  \, x^3 \ \tilde w'( x)  \tilde w( x) \,, \qquad 
  \mathcal{I}_6  &=  \int  \textrm{d} x  \, x^2 \  \tilde w'( x) \tilde w( x)   \,,
\end{align}
so that we can easily express
\begin{align}
  4 \left \langle E^a_{i} E^b_{j} \right \rangle \left \langle B^a_{i} B^b_{j} \right \rangle_1   = & \frac{\Omega_1 e^{4 \pi (\kappa - \mu)}}{ (2 \pi)^6} H^8  \mathcal{I}_1 \mathcal{I}_2  \, , \quad 
  && 4 \left \langle E^a_{i} E^b_{j} \right \rangle \left \langle B^a_{i} B^b_{j} \right \rangle_2   =  \frac{\Omega_1 e^{4 \pi (\kappa - \mu)}}{ (2 \pi)^6} H^8 \xi^2  \mathcal{I}_1 \mathcal{I}_3  \, , \nonumber  \\
  4 \left \langle E^a_{i} E^b_{j} \right \rangle \left \langle B^a_{i} B^b_{j} \right \rangle_3  = & \frac{\Omega_1 e^{4 \pi (\kappa - \mu)}}{ (2 \pi)^6} H^8 (-\xi) \mathcal{I}_1  \mathcal{I}_4  \, , \quad
  && 4 \left \langle E^a_{i} E^b_{j} \right \rangle \left \langle B^a_{i} B^b_{j} \right \rangle_4   =  \frac{\Omega_1 e^{4 \pi (\kappa - \mu)}}{ (2 \pi)^6} H^8 (-\xi)  \mathcal{I}_1  \mathcal{I}_4  \, , \nonumber \\
  4 \left \langle E^a_{i} B^b_{j} \right \rangle_1 \left \langle B^a_{i} E^b_{j} \right \rangle_1   = & \frac{\Omega_1 e^{4 \pi (\kappa - \mu)}}{ (2 \pi)^6} H^8  \mathcal{I}_5^2  \, , \quad 
  && 4 \left \langle E^a_{i} B^b_{j} \right \rangle_1 \left \langle B^a_{i} E^b_{j} \right \rangle_2   =  \frac{\Omega_1 e^{4 \pi (\kappa - \mu)}}{ (2 \pi)^6} H^8 (-\xi)  \mathcal{I}_5  \mathcal{I}_6  \, , \nonumber \\
  4 \left \langle E^a_{i} B^b_{j} \right \rangle_2 \left \langle B^a_{i} E^b_{j} \right \rangle_1   = & \frac{\Omega_1 e^{4 \pi (\kappa - \mu)}}{ (2 \pi)^6} H^8 (-\xi)  \mathcal{I}_5 \mathcal{I}_6  \, , \quad
  && 4 \left \langle E^a_{i} B^b_{j} \right \rangle_2 \left \langle B^a_{i} E^b_{j} \right \rangle_2   =  \frac{\Omega_1 e^{4 \pi (\kappa - \mu)}}{ (2 \pi)^6} H^8 \xi^2  \mathcal{I}_6^2 \, , \nonumber \\
  4 \left \langle E^a_{i} B^b_{j} \right \rangle_3 \left \langle B^a_{i} E^b_{j} \right \rangle_3   = & \frac{\Omega_2 e^{4 \pi (\kappa - \mu)}}{ (2 \pi)^6} H^8 \xi^2 \mathcal{I}_3^2 \, , 
\end{align}
where we have also used $ef = -\xi/\tau$ and $\textrm{d}k k^2 = - \textrm{d}x_k x_k^2 /\tau^3 $. At this point we have all in hand to compute $ T_0 $, $ T_1 $ and $ T_2 $. Let us start with $T_0$:
\begin{equation}
  T_0 = \left \langle E^a_{i} B^b_{j} \right \rangle_1 + \left \langle E^a_{i} B^b_{j} \right \rangle_2 + \left \langle E^a_{i} B^b_{j} \right \rangle_3 = - \frac{H^4}{4 \pi^2} e^{2 \pi (\kappa - \mu)} \,  \left( \mathcal{I}_5 + \frac{\xi}{2} \mathcal{I}_3 - \xi \mathcal{I}_6 \right) \, ,
\end{equation}
which to a good approximation is given by
\begin{equation}
  T_0 = - 0.24 \times \frac{H^4}{4 \pi^2} \times \xi^3  e^{2 \pi (\kappa - \mu)}  \equiv   \frac{H^4}{4 \pi^2} \, e^{2 \pi (\kappa - \mu)}  \, \tilde T_0 \, .
\end{equation}
Analogously we can compute $T_1$ and $T_2$
\begin{align}
  T_1 & = \frac{\Omega_1 H^8}{ 4 (2\pi)^6} e^{4 \pi (\kappa - \mu)} \  \mathcal{I}_1\left[ \mathcal{I}_2  + \xi^2 \mathcal{I}_3 - 2 \xi \mathcal{I}_4 \right]  \,, \\
T_2 & = \frac{\Omega_1 H^8}{ 4 (2\pi)^6} e^{4 \pi (\kappa - \mu)} \ \left[\mathcal{I}_5^2 - 2 \xi \mathcal{I}_5 \mathcal{I}_6+ \xi^2 \mathcal{I}_6^2 + \xi^2 \Omega_2/\Omega_1 \mathcal{I}_3^2 \right]  \, .
\end{align}
Performing the integrals yields
\begin{align}
  T_1 & \simeq 0.0082 \times \frac{H^8}{80 \pi^4} \times \xi^8  e^{4 \pi (\kappa - \mu)}  \equiv  \frac{H^8}{80 \pi^4} \,  e^{4 \pi (\kappa - \mu)} \, \tilde T_1  \, , \\
  T_2 & \simeq 0.051 \times \frac{H^8}{80 \pi^4} \times  \xi^6 e^{4 \pi (\kappa - \mu)} \equiv  \frac{H^8}{80 \pi^4} \,  e^{4 \pi (\kappa - \mu)} \, \tilde T_2  \, ,
\end{align}
where we have also substituted the values of $\Omega_1$ and $\Omega_2$. 
