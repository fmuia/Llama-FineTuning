In this section\footnote{To ease the notation, in Secs.~\ref{sec:non-minimal} and~\ref{sec:attractors} we will work in Planck units, $m_p = 1$, while reintroducing $m_p$ in Sec.~\ref{sec:PBHs}.} we generalize the analysis presented in Sec.~\ref{sec:review} to the case where the inflaton is non-minimally coupled to gravity. In general non-minimal couplings between the inflaton and gravity may naturally arise in the context of supergravity and string theory~\cite{Kallosh:2000ve} or from radiative corrections in the framework of QFT in curved space-time.\footnote{For an introduction to the topic see for example~\cite{Birrel:1984}.} Moreover, during the last years it was realized that the introduction of a non-minimal coupling between the inflaton and gravity (in the Jordan frame) can provide a mechanism to ``flatten'' the inflationary potential (in the Einstein frame)~\cite{Bezrukov:2007ep,Bezrukov:2009db,Kallosh:2013tua,Einhorn:2009bh,Ferrara:2010yw,Buchmuller:2013zfa,Giudice:2014toa,Pallis:2013yda,Pallis:2014dma,Pallis:2014boa,Ellis:2013xoa,Kallosh:2013xya,
Nakayama:2010ga} leading to predictions for the scalar spectral index $n_s$ and the  tensor-to-scalar-ratio $r$  that lie right in the sweet spot of the constraints set by CMB observations~\cite{Ade:2015lrj}. For these reasons it is interesting to consider a generalization of the mechanism discussed in Sec.~\ref{sec:review} in order to take these models into account.

 
We start by considering the action for a pseudoscalar field $\phi$ non-minimally coupled to gravity and coupled to an Abelian gauge field\footnote{The generalization to $\mathcal{N}$ Abelian gauge fields $A^a_{\mu}$ is straight-forward. To ease the notation, we will work with ${\cal N} = 1$ here and re-introduce the parameter ${\cal N}$ only in the final expressions.} through the topological Chern-Simons term,
\begin{equation}
	\label{non_minimal:jordan_frame}
	\mathcal{S} = \int \textrm{d}^4 x \sqrt{-g_J} \left[ \frac{\, \Omega(\phi ) \, R}{2} -X -V_J -\frac{g_J^{\mu\rho}g_J^{\nu\sigma}}{4} F_{\mu\nu}F_{\rho\sigma} -\frac{\alpha}{8 \Lambda} \frac{\varepsilon^{\mu\nu\rho\sigma}}{ \sqrt{-g_J}} \, \phi \, F_{\mu\nu}F_{\rho\sigma} \right] \ ,
\end{equation}
where $\varepsilon^{\mu\nu\rho\sigma}$ is the Levi-Civita symbol, $X \equiv g_J^{\mu \nu} \partial_{\mu} \phi \partial_{\nu} \phi /2$, $F_{\mu\nu} \equiv \partial_{\mu} A_{\nu} - \partial_{\nu} A_{\mu}$, $\alpha$ is the coupling between the pseudoscalar and the gauge field, $\Lambda $ is a mass scale that suppresses the higher dimensional operator $\phi F \tilde{F}$ and $\Omega(\phi)$ is function of $\phi$ that without loss of generality can be expressed as:
\begin{equation}
	\Omega(\phi) = 1 + \varsigma f(\phi) \ , 
\end{equation}
where $\varsigma$ is a the dimensionless coupling constant and $f(\phi)$ is a generic function of $\phi$. The case discussed in Sec.~\ref{sec:review} can be easily recovered by setting $\varsigma = 0$.

The action of Eq.~\eqref{non_minimal:jordan_frame} corresponds to a Jordan frame formulation of the theory (explaining the subscript J). The Einstein frame formulation (where gravity is described by a standard Einstein-Hilbert term) of the theory is found via a conformal transformation:
\begin{equation}
	g_{J\, \mu\nu} \rightarrow g_{\mu\nu} = \Omega(\phi) g_{J\, \mu\nu} \, ,
\end{equation}
so that the action reads:
\begin{equation}
	\label{non_minimal:einstein_frame}
	\mathcal{S} = \int \textrm{d}^4 x \sqrt{-g} \left[ \frac{R}{2} - K(\phi)X -V_E -\frac{g^{\mu\rho}g^{\nu\sigma}}{4} F_{\mu\nu}F_{\rho\sigma} -\frac{\alpha}{4 \Lambda} \frac{\varepsilon^{\mu\nu\rho\sigma}}{2 \sqrt{-g}} \, \phi \, F_{\mu\nu}F_{\rho\sigma} \right] \ ,
\end{equation}
where $V_E \equiv V_J/\Omega^2(\phi)$ and $K(\phi)$ is defined as:\footnote{In order to make expressions simpler sometimes we will write $K$ instead of $K(\phi)$.}
\begin{equation}
 	K(\phi) \equiv \Omega^{-1} +  \frac{3}{2} \left( \frac{\textrm{d} \ln \Omega }{\textrm{d} \phi} \right)^2 \, .
 \end{equation} 
Note that the action shown in Eq.~\eqref{non_minimal:einstein_frame} is the action for a pseudoscalar field that is minimally coupled to gravity but that has a non-standard kinetic term.\footnote{Pseudoscalar theories from string theory compactifications typically exhibit a canonically coupling to gravity, a coupling of the pseudoscalar to the topological term of a gauge theory as well as a non-canonical kinetic term for the pseudoscalar. It is hence in the formulation in Eq.~\eqref{non_minimal:einstein_frame} that our model may arise from string theory. For more details see appendix \ref{app:axions}.} As a consequence the treatment carried out in the rest of this section is valid for both these cases.


In the following we discuss the evolution of the system defined by the action of Eq.~\eqref{non_minimal:einstein_frame}. In particular we study both the evolution of the homogeneous background and of perturbations around this background. As a first step we start by computing the equation of motion for the gauge field. For this purpose we take the variation of the action (expressed in terms of conformal coordinates) with respect to $A_{\nu}$:
\begin{equation}
\frac{\delta \mathcal{S}}{\delta A_{\nu}}  = \partial_{\mu} \left[ \frac{\eta^{\mu\rho}\eta^{\nu\sigma}}{2}\left( \partial_{\rho} A_{\sigma} - \partial_{\sigma} A_{\rho} \right) \right] + \frac{\alpha}{4 \Lambda} \partial_{\mu} \left[ \phi \, \varepsilon^{\mu\nu\rho\sigma} \left( \partial_{\rho} A_{\sigma} - \partial_{\sigma} A_{\rho} \right) \right] = 0 \ ,
\end{equation}
where $\eta_{\mu \nu}$ is the Minkowski metric. At this point we can proceed by fixing the gauge. A convenient gauge to describe the problem is Coulomb gauge, i.e.\ $g^{\mu \nu }\partial_{\mu} A_{\nu} = 0 , A_0 = 0$ so that the equation of motion reads:
\begin{equation}
\eta^{\nu\sigma} \Box A_{\sigma} - \frac{\alpha}{\Lambda} \phi^{\prime} \varepsilon^{0\nu\rho\sigma} \partial_{\rho} A_{\sigma} = 0 \ , 
\end{equation}
where $\Box$ is the standard flat space d'Alembert operator $\Box \equiv \eta^{\mu\nu} \partial_{\mu} \partial_{\nu}$. As a consequence we can write:
\begin{equation}
\label{non_minimal:gauge_fields}
\Box \vec{A}  - \frac{\alpha}{\Lambda} \phi^{\prime} \, \vec{\nabla} \times \vec{A} = 0 \ , 
\end{equation}
where $\vec{\nabla} \times $ is the usual curl operator and a prime is used to denote a derivative with respect to $\tau$. This equation matches with the Eq.~\eqref{review:eq_motionA} and thus the solution for the gauge fields is exactly the same as in the case of a minimal coupling to gravity.

We can now consider the variation of the action (again in conformal time) with respect to $\phi$ to get the equation of motion for the inflaton:
\begin{equation}
	\partial_\mu \left( \eta^{\mu \nu} K(\phi) a^2 \, \partial_\nu \phi \right) -a^4 \left( \frac{\partial V_E}{\partial \phi}(\phi) + \frac{\alpha}{4 \Lambda} F^{\mu\nu}\tilde{F}^{\mu\nu}\right) = 0 \ ,
\end{equation}
which can be expressed as:
\begin{equation}
	\label{non_minimal:eom_phi_general}
	- K(\phi) \Box\phi -\frac{K_{,\phi}(\phi)}{2} \eta^{\mu\nu} \, \partial_\mu \, \phi \partial_\nu \phi+ 2a \mathcal{H} K(\phi)\phi^\prime + a^2 \, V_{E,\phi}(\phi) + a^2 \frac{\alpha}{\Lambda} \langle \vec{E}\cdot \vec{B}\rangle = 0  \, ,
\end{equation}
where ${}_{,\phi} $ is used to denote a derivative with respect to $\phi$, primes are used to denote a derivatives with respect to $\tau$ and $\mathcal{H}\equiv a^\prime/a$. At this point we can express $\phi$ as:
\begin{equation}
	\phi(\tau,\vec{x}) = \phi_0(\tau) + \delta \phi(\tau,\vec{x}) \ , 
\end{equation}
where $\phi_0(\tau)$ is the (homogeneous) background solution and $\delta \phi(\tau,\vec{x})$ is a small perturbation. By substituting this parametrization into Eq.~\eqref{non_minimal:eom_phi_general} it is easy to show that the equation of motion for the background field $\phi_0$ reads:
\begin{equation}
	\label{non_minimal:background_conformal}
	K(\phi_0) \phi_0^{\prime \prime} + \frac{K_{,\phi}(\phi_0)}{2} \phi_0^{\prime\, 2} + 2a \mathcal{H} K(\phi_0)\phi_0^\prime + a^2 \, V_{E,\phi}(\phi_0) + a^2 \frac{\alpha}{\Lambda} \langle \vec{E}\cdot \vec{B}\rangle = 0  \, .
\end{equation}
Conversely, the linearized equation of motion for the perturbation reads:
\begin{equation}
\begin{aligned}
	\label{non_minimal:perturbations}
	&\delta\phi^{\prime \prime} - \vec{\nabla}^2\delta\phi +\frac{1}{2}\frac{\textrm{d}}{\textrm{d} \phi} \left(\frac{\textrm{d} \ln K}{\textrm{d} \phi}\right) \phi_0^{\prime\, 2} \delta\phi + \frac{\textrm{d} \ln K}{\textrm{d} \phi} \phi_0^{\prime} \delta\phi^\prime + 2a \mathcal{H} \delta \phi^\prime +  \\ 
	&+ a^2 \frac{\textrm{d}}{\textrm{d} \phi} \left( \frac{V_{E,\phi}}{K}\right)\delta \phi  - \frac{a^2}{K^2} K_{,\phi} \frac{\alpha}{\Lambda} \langle \vec{E}\cdot \vec{B}\rangle \delta \phi + \frac{a^2}{K} \frac{\alpha}{\Lambda} \delta\left( \langle \vec{E}\cdot \vec{B}\rangle\right) = 0 \ .
\end{aligned}
\end{equation}
A detailed analysis on the evolution of the background and of the inflaton perturbation is presented in Sec.~\ref{sec:background} and in Sec.~\ref{sec:perturbations} respectively. 

\subsection{Background evolution}
\label{sec:background}
As a first step we express Eq.~\eqref{non_minimal:background_conformal} in terms of cosmic time ($\textrm{d}t = a \, \textrm{d}\tau$) and we divide by $K(\phi_0)$ to get:
\begin{equation}
	\label{non_minimal:background_cosmic}
	\ddot{\phi}_0 + \frac{1}{2} \frac{\textrm{d} \ln K}{\textrm{d} \phi_0} \dot{\phi}_0^2 + 3 H \dot{\phi}_0 + \frac{V_{E,\phi_0}}{K} + \frac{\alpha}{\Lambda K} \langle \vec{E}\cdot \vec{B}\rangle = 0 \ .
\end{equation}
We can also express Friedmann equation as:
\begin{equation}
	\label{non_minimal:background_friedmann}
	3 H^2 = K \frac{\dot{\phi}_0^2}{2} + V_E + \frac{1}{2} \langle \vec{E}^2 + \vec{B}^2 \rangle \ .
\end{equation}
At this point we can solve Eq.~\eqref{non_minimal:gauge_fields} and plug the solution into Eq.~\eqref{non_minimal:background_cosmic} and Eq.~\eqref{non_minimal:background_friedmann} to get:\footnote{Notice that these expressions for $\langle \vec{E}\cdot \vec{B}\rangle$ and $\langle \vec{E}^2 + \vec{B}^2 \rangle$ only hold for sizable values for $\xi$ (i.e.\ for $\xi \gtrsim 1$). Different expressions should be used for small values of $\xi$, see~\cite{Anber:2009ua}.}
\begin{eqnarray}
	\label{non_minimal:background_scalar}
	\ddot{\phi}_0 + \frac{1}{2} \frac{\textrm{d} \ln K}{\textrm{d} \phi_0} \dot{\phi}_0^2 + 3 H \dot{\phi}_0 + \frac{V_{E,\phi_0}}{K} - 2.4 \cdot 10^{-4} \, \frac{\alpha}{\Lambda K} \frac{H^4}{\xi^4} e^{2 \pi \xi} = 0 \ , \\
	\label{non_minimal:background_friedmann_2}
	3 H^2 = K \frac{\dot{\phi}_0^2}{2} + V_E + 1.4 \cdot 10^{-4} \frac{H^4}{\xi^3} e^{2 \pi \xi}	 \ ,
\end{eqnarray}
with the parameter $\xi$ defined as in Eq.~\eqref{review:xi},
\begin{equation}
	\label{non_minimal:xi_def}
	\xi \equiv \frac{\alpha}{2 \Lambda} \, \frac{|\dot{\phi}_0|}{H} \ .
\end{equation}
Note that here $\phi_0$ is not the canonically normalized field. If the field $\phi_0$ is slow-rolling these equations can be approximated as:
\begin{eqnarray}
	\label{non_minimal:slow_roll_gauge}
	 3 H \dot{\phi}_0 + \frac{V_{E,\phi_0}}{K} - 2.4 \cdot 10^{-4} \, \frac{\alpha}{\Lambda K} \frac{H^4}{\xi^4} e^{2 \pi \xi} \simeq 0 \ , \\
	 	\label{non_minimal:slow_roll_friedmann}
	3 H^2 \simeq V_E + 1.4 \cdot 10^{-4} \frac{H^4}{\xi^3} e^{2 \pi \xi} \ ,
\end{eqnarray}
so that in the first part of the evolution (where $\xi \simeq 1$) we recover the usual slow-roll equations:
\begin{equation}
	\label{non_minimal:slow_roll}
	\dot{\phi}_0 \simeq - \frac{V_{E,\phi_0}}{3 H K} \ , \qquad \qquad 3 H^2 \simeq V_E \ .
\end{equation}
As in Sec.~\ref{sec:review}, the back-reaction of the gauge fields on the Friedmann equation (i.e.\ Eq.~\eqref{non_minimal:slow_roll_friedmann}) is negligible throughout the whole evolution, but it may have a significant impact on Eq.~\eqref{non_minimal:slow_roll_gauge}. As above, the $\langle \vec{E}\cdot \vec{B} \rangle$ term plays the role of a friction term which grows exponentially with $\xi$. Notice that as long as the gauge field contribution is not too strong we find for the first slow-roll parameter $\epsilon_H$,\footnote{As long as the back-reaction of the gauge fields is negligible, this definition is equivalent to the more commonly used `potential' slow-roll parameter, $\epsilon_H \simeq \epsilon_V = m_p^2 (V'(\phi)/V)^2/(2 K)$.}
\begin{equation}
	\label{non_minimal:epsilon_H}
	\epsilon_H \equiv -\frac{\dot{H}}{H^2} \simeq \frac{1}{2} K(\phi_0) \left( \frac{\dot{\phi}_0}{H} \right)^2 \propto K(\phi_0) \xi^2 \ ,
\end{equation}
where we have used Eq.~\eqref{non_minimal:slow_roll} and Eq.~\eqref{non_minimal:xi_def}. As for a vast majority of slow-roll inflationary models $\epsilon_H$ grows during inflation,\footnote{In particular this happens for 
all the models considered in Sec.~\ref{sec:attractors} (see Fig.~\ref{fig:potentials_and_xi}).} for sufficiently large values of $\alpha/\Lambda$ we expect a strong gauge field production that strongly affects the dynamics towards the end of inflation. However, it is crucial to stress that in general $\xi$ is not \emph{monotonically} increasing for all the slow-roll inflationary models. For example, if both  $\epsilon_H$ and $K(\phi_0)$ monotonically increase during inflation, their interplay may lead to a local maximum in $\xi$, if the growth in $K(\phi_0)$ dominates the growth of $\epsilon_H$ at late times. Such a scenario will be particularly relevant for the discussion of PBHs in Sec.~\ref{sec:PBHs}. 

\subsection{Perturbations}
\label{sec:perturbations}
In order to study the evolution of the inflaton perturbations we start by expressing the $\delta\left( \langle \vec{E}\cdot \vec{B}\rangle\right)$ term appearing in Eq.~\eqref{non_minimal:perturbations} in a convenient form. In particular following the treatment of~\cite{Anber:2009ua} we proceed by using:
\begin{equation}
\label{non_minimal:gauge_field_contrib}
\delta\left( \langle \vec{E}\cdot \vec{B}\rangle\right) = [\vec{E} \cdot \vec{B} - \langle \vec{E} \cdot \vec{B} \rangle]_{\delta \phi = 0} + \frac{\partial \langle \vec{E} \cdot \vec{B} \rangle}{\partial \dot{\phi}_0}\frac{ \delta \phi^{\prime} }{a} \equiv \delta_{\vec{E}\cdot \vec{B}} \, -  \left( \frac{ \pi \alpha}{ \Lambda H } \right)  \langle \vec{E} \cdot \vec{B} \rangle \, \frac{\delta \phi^\prime}{a} \ .
\end{equation}
We can thus proceed by substituting this expression into Eq.~\eqref{non_minimal:perturbations} (expressed in terms of cosmic time) to get:
\begin{equation}
\label{non_minimal:scalar_fluctuations}
\begin{aligned}
	&\ddot{\delta\phi} - \frac{\vec{\nabla}^2}{a^2}\delta\phi +\frac{1}{2} \frac{\textrm{d}}{\textrm{d} \phi_0} \left(\frac{\textrm{d} \ln K}{\textrm{d} \phi_0}\right) \dot{\phi}_0^{ 2} \delta\phi + \frac{\textrm{d} \ln K}{\textrm{d} \phi_0} \dot{\phi}_0 \dot{\delta\phi} + 3H \dot{\delta \phi} + \frac{\textrm{d}}{\textrm{d} \phi_0} \left( \frac{V_{E,\phi_0}}{K}\right)\delta \phi \\ 
	& - \frac{1}{K} \frac{\textrm{d} \ln K}{\textrm{d} \phi_0}  \frac{\alpha}{\Lambda} \langle \vec{E}\cdot \vec{B}\rangle \delta \phi + \frac{1}{K} \frac{\alpha}{\Lambda} \left[ \delta_{\vec{E}\cdot \vec{B}} \, -  \left( \frac{ \pi \alpha}{ \Lambda H } \right)  \langle \vec{E} \cdot \vec{B} \rangle \dot{\delta \phi} \right] = 0 \ .
\end{aligned}
\end{equation}
Let us start by discussing the regime where the gauge field contribution is negligible before we proceed by considering the strong gauge field regime. If the gauge field contribution is negligible the scalar power spectrum $\Delta^2_s(k)$ is simply given by the standard vacuum amplitude:
\begin{equation}
	\left.   \Delta^2_s (k) \right|_\text{vac} \simeq   \frac{ 1 }{8 \, \pi^2 } \frac{H^2}{  c_s \ \epsilon_H }\ , 
\end{equation}
where $\epsilon_H$ is the first (Hubble) slow-roll parameter defined in Eq.~\eqref{non_minimal:epsilon_H} and $c_s$ is the speed of sound that for all the models considered in this work is equal to one. Using Eq.~\eqref{non_minimal:slow_roll} it is easy to show that the vacuum amplitude can be expressed as
\begin{equation}
	\label{non_minimal:vacuum}
	\left.   \Delta^2_s (k) \right|_\text{vac} \simeq   \frac{ K V }{12 \, \pi^2 }  \left( \frac{\textrm{d} \ln V_E}{\textrm{d} \phi_0}\right)^{-2} \ .
\end{equation}
Conversely, in the strong gauge field regime the gauge-field induced contributions (i.e.\ the terms proportional to $\langle \vec{E}\cdot\vec{B} \rangle $) are large. At this point we can neglect all the higher order terms in the slow-roll parameters as well as all the terms that are not exponentially enhanced\footnote{For more details on the derivation of this formula and for consistency checks see appendix~\ref{sec:spectrum}.} and the scalar power spectrum in the strong gauge field regime can be expressed as:
 \begin{equation}
 	\label{non_minimal:gauge_sourced}
 	\left.  \Delta^2_s (k) \right|_\text{gauge} \simeq  \left(\frac{\alpha \langle \vec{E}^a \cdot \vec{B}^a  \rangle / \sqrt{\mathcal{N}} }{3 \, b \, \Lambda \, \dot{\phi}_0 \, H \, K }\right)^2  \ ,
 \end{equation}
  where we have defined:
\begin{equation}
	b \equiv  1  -  \pi \left( \frac{\alpha}{\Lambda} \right)^2  \frac{\langle \vec{E}^a \cdot \vec{B}^a  \rangle}{3 \, H^2 K} \ ,
	\label{eq:b}
\end{equation}
and we have generalized the result to the case where $\mathcal{N}$ Abelian gauge fields are coupled to the inflaton. Finally we can use both Eq.~\eqref{non_minimal:vacuum} and Eq.~\eqref{non_minimal:gauge_sourced} to get the complete expression for the scalar power spectrum:
 \begin{equation}
 	\label{non_minimal:scalar_final}
 	\Delta^2_s (k) \simeq  \frac{ 1 }{K}  \left( \frac{H^2 }{2 \pi |\dot{\phi}_0|}\right)^{2} + \left(\frac{\alpha \langle \vec{E}^a \cdot \vec{B}^a  \rangle / \sqrt{\mathcal{N}} }{3 \, b \, \Lambda \, \dot{\phi}_0 \, H \, K }\right)^2 \ .
 \end{equation}
In the regime where the gauge fields dominate the evolution we can approximate $b$ by simply neglecting the constant term (i.e.\ the first term in Eq.~\eqref{eq:b}). One can then easily show that:
\begin{equation}
	\label{non_minimal:scalar_strong_approx}
 	\left.  \Delta^2_s (k) \right|_{\text{gauge}} \simeq  \frac{  1 }{ 4 \pi^2 \xi^2 \mathcal{N} }  \ ,
 \end{equation}
which corresponds (at least in form) to the expression given in Eq.~\eqref{review:scalar_strong}. There are some conceptual (and indeed physical) differences that must be pointed out:
\begin{itemize}
	\item While Eq.~\eqref{review:scalar_strong} is expressed in terms of a canonically normalized field, Eq.~\eqref{non_minimal:scalar_strong_approx} is expressed in terms of the non-canonically normalized field. As a consequence, in order to match the two equations we should express the dynamics in terms of a canonically normalized field $\varphi$ defined as $(\textrm{d} \varphi / \textrm{d} \phi_0)^2 \equiv K(\phi_0)$. Using this definition Eq.~\eqref{non_minimal:scalar_strong_approx} can be turned into:
	\begin{equation}
	\label{non_minimal:scalar_strong_approx_varphi}
 	\left.  \Delta^2_s (k) \right|_\text{gauge} \simeq  \frac{  K(\phi_0) }{ 4 \pi^2 \mathcal{N}   \tilde{\xi}^2 }  \propto \frac{K}{\epsilon_H} ,
 \end{equation}
	where $\tilde{\xi} \equiv \alpha |\dot{\varphi}| / (2 \Lambda H )$. The difference between Eq.~\eqref{non_minimal:scalar_strong_approx_varphi} and Eq.~\eqref{review:scalar_strong} is now obvious: in terms of the canonical field $\varphi$ in the Einstein frame, the introduction of a non-minimal coupling to gravity implies that the coupling to the gauge fields is altered as $\dot \varphi \mapsto \dot \varphi/\sqrt{K}$ and hence all effects sourced by the gauge fields are suppressed as $K$ is increased. This can be traced back to the choice of coupling the Chern-Simons term to the canonically normalized field $\phi_0$ in Jordan frame in Eq.~\eqref{non_minimal:jordan_frame}.
	
	\item In the minimally coupled case with monotonically increasing $\epsilon_H$, Eq.~\eqref{review:scalar_strong} holds from some critical value until the end of inflation. Now, if $K(\phi_0)$ strongly increases towards the end of inflation, a monotonic growth of $\epsilon_H$ no longer implies a monotonic growth of $\xi \propto \sqrt{\epsilon_H/K(\phi_0)}$, see Eq.~\eqref{non_minimal:epsilon_H}. Consequently, $\xi$ can be strongly suppressed towards the end of inflation, shutting off the tachyonic instability for the gauge fields. In particular after the usual nearly scale invariant power spectrum of slow-roll inflation at CMB scales and the following gauge-field induced increase, we can achieve another regime with $ b \simeq 1$ and thus $\Delta^2_s (k)$ is much smaller than the prediction of Eq.~\eqref{non_minimal:scalar_strong_approx}.\footnote{Note that Eq.~\eqref{non_minimal:scalar_strong_approx} only holds for $b \gg 1$. For $b \simeq 1$, a reduction in $\xi$ leads to an (exponential) reduction in the scalar power spectrum.} In this setup, the scalar power spectrum thus features a bump. As discussed in~\cite{Clesse:2015wea} such a particular shape for the spectrum can be extremely interesting for the generation of PBHs. A detailed discussion of this topic is carried out in Sec.~\ref{sec:PBHs}.

	\item The analysis performed here is complementary to the proposal of Ref.~\cite{Garcia-Bellido:2016dkw}, where a bump in the scalar spectrum (and hence in the PBH spectrum) is achieved based on the action~\eqref{review:action_pseudoscalar}. Instead of introducing a non-minimal coupling to gravity as we do here, they instead modify the scalar potential so that the velocity of the inflaton (and hence the parameter $\xi$) is reduced in the last stages of inflation. On the contrary, after a conformal transformation to the Einstein frame and a canonical normalization of the inflaton field, our scalar potential is essentially featureless. The non-trivial evolution of $\xi$ in this frame is sourced by the coupling between the inflaton and gauge fields.
\end{itemize}

Before concluding this section briefly discuss the spectrum $\Delta_t^2$ of GWs that are generated in these generalized models of inflation. In order to compute the shape of the spectrum we  start once again by considering the linearized Einstein equation. In particular, it can be expressed as in Eq.~\eqref{review:linearized_einstein}, where $T_{\mu\nu}$ only depends on the gauge fields.\footnote{Notice that $\sqrt{-g} \phi F_{\mu \nu} \tilde{F}^{\mu \nu} $ does not depend on $g_{\mu \nu}$ and $h^{ij}\partial_i \phi \partial_j \phi = 0$ at first order.} Moreover, as \begin{equation}
 \sqrt{-g_J} \, g_J^{\mu \rho} g_J^{\nu \sigma} F_{\mu \nu} F_{\rho \sigma} = \sqrt{-g} \, g^{\mu \rho} g^{\nu \sigma} F_{\mu \nu} F_{\rho \sigma}  \ , 
\end{equation}
the source term for GWs is not affected by the presence of a non-minimal coupling between the inflaton and gravity. As a consequence the spectrum of GWs is once again given by Eq.~\eqref{review:OmegaGW}. However, it is crucial to stress that while the expression of the spectrum has exactly the dependence on $\xi$ as in Sec.~\ref{sec:review}, the evolution of $\xi$ is now expected to be different and thus (consistently with the examples shown in Sec.~\ref{sec:attractors}) we expect the spectrum to be different.