\usepackage[T1]{fontenc}
\usepackage{hyperref}
\usepackage{geometry,amsmath,amsfonts,amssymb,amsthm}
\usepackage{slashed}
\usepackage{epsfig}
\usepackage{latexsym}
\usepackage{graphicx}
\usepackage{amssymb}
\usepackage{array}
\usepackage{verbatim}
\usepackage{bm}
\usepackage{lmodern}
\usepackage{dsfont}
\usepackage[footnotesize]{caption}
\usepackage{subfigure}
\usepackage{diagbox}

\usepackage{cancel}
\usepackage{cite}

\usepackage{float}

\usepackage{multirow}
\usepackage{tikz}

\usepackage{bbold}


\usepackage{rotating}
\usepackage{ifthen}
\usepackage[utf8]{inputenc}  


\setcounter{section}{0}
\numberwithin{equation}{section} 
\setlength\parindent{0pt} 

\usepackage{varioref}
\usepackage{prettyref}

\usepackage{microtype}



\newrefformat{fig}{Fig.~\ref{#1}}
\newrefformat{subsec}{Sec.~\ref{#1}}
\newrefformat{eq}{Eq.~\eqref{#1}}
\newrefformat{fn}{Footnote~\ref{#1}}
\newrefformat{app}{App.~\ref{#1}}
\newrefformat{conj}{Conjecture~\ref{#1}}
\newrefformat{lem}{Lemma~\ref{#1}}
\newrefformat{exa}{Example~\ref{#1}}
\newrefformat{def}{Definition~\ref{#1}}
\newrefformat{thm}{Theorem~\ref{#1}}
\newrefformat{tab}{Table~\ref{#1}}

\usepackage{textcomp}

\renewcommand\[{\begin{equation}}
\renewcommand\]{\end{equation}}
\renewenvironment{align*}{\align}{\endalign}

\newtheorem{thm}{Theorem}
\newtheorem{lem}[thm]{Lemma}
\newtheorem{conjecture}[thm]{Conjecture}
\newtheorem{example}[thm]{Example}
\newtheorem{defn}[thm]{Definition}


\hypersetup{pdfencoding=auto, psdextra}

\pdfstringdefDisableCommands{%
  \def\to{\textrightarrow}%
}
\pdfsuppresswarningpagegroup=1

\hypersetup{ pdfborder={0 0 .7 [1 2]} }



\geometry{verbose,tmargin=2.5cm,bmargin=2.5cm,lmargin=2cm,rmargin=2cm,marginparwidth=1.6cm}

 \edef\oldeverydisplay{\the\everydisplay}
\usepackage[nodisplayskipstretch]{setspace}
\everydisplay\expandafter{\oldeverydisplay}
\setstretch{1.3}
\setlength{\footnotesep}{0.7\baselineskip}

\newcommand{\Kahler}{\ensuremath{\text{K}\ddot{\text{a}}\text{hler}\,}}
\newcommand{\dAlemb}{\Box} 

\DeclareMathOperator{\PR}{PR}
\DeclareMathOperator{\NR}{NR}
\DeclareMathOperator{\sn}{sn}
\DeclareMathOperator{\am}{am}



