Axion-like particles are one of the most natural candidates to drive cosmic inflation since their perturbative shift-symmetry protects the flatness of their potential against quantum corrections. These pseudoscalar fields typically couple to the gauge fields present in the theory through a Chern-Simons term in the Lagrangian. As reviewed in Sec.~\ref{sec:review}, in models of axion-inflation, such a coupling can lead to several interesting and possibly observable phenomenological features. In particular, the production of chiral gravitational waves is a peculiar signature of these models that could be observed in the near future in experiments like the ground-based interferometer LIGO or the space-mission LISA. Hence, it is extremely interesting to continue exploring the rich phenomenology provided by these inflationary models. In this paper we showed that if the axion-inflaton is non-minimally coupled to gravity, these models of axion-inflation represent a natural framework for the production of PBH dark matter. In particular, in Sec.~\ref{sec:non-minimal} we worked out the main generic features of an axion-inflation model with non-minimal coupling to gravity, deriving both the background dynamics and the perturbation power spectra. As explained in Sec.~\ref{sec:attractors}, the interplay between the two main features of this model, i.e. the coupling to gauge fields and the non-minimal coupling to gravity, makes the generation of a broad peak in the scalar spectrum possible. In particular, the increase of the spectrum is sourced by the instability caused by the coupling between the inflaton and the gauge fields. If the kinetic function $K(\phi)$ (see Eq.~\ref{non_minimal:einstein_frame}), corresponding to a non-minimal coupling to gravity in the Jordan frame, features a rapid increase after the advent of this instability, a subsequent suppression of the spectrum arises, generating a peak as shown in Fig.~\ref{fig:BH2}. The position of the peak is determined by the choice of $K(\phi)$ and by the ratio $\alpha/\Lambda$ (see Eq.~\ref{non_minimal:einstein_frame}), which parametrizes the strength of the inflaton coupling to gauge fields. In this context, an interesting class of models is given by the \textit{attractors at strong coupling}, defined in Eq.~\eqref{attractors:model_def}. As shown in Sec.~\ref{sec:PBHs}, by choosing the kinetic function $K(\phi)$ in such a way that the previously mentioned condition is met (see Eq.~\eqref{attractors:inv_definition}), PBHs produced with this mechanism can account for a sizeable fraction of the dark matter present in the universe in a wide region of the parameter space. As explained in the main text, if we ignore the disputable bounds based on neutron star capture, PBHs can even account for the totality of dark matter, in the mass range $10^{18}$~g $\, - \, 10^{24}$~g. Heavier PBHs with masses around a solar mass, although accounting only for a fraction of dark matter, may in turn lead to observable GW signals in LIGO.

The inflationary model employed and the PBH production mechanism proposed in this paper can be further developed following several complementary directions. For instance, they should be embedded in a consistent cosmological scenario, which includes the reheating of the Standard Model degrees of freedom. Since the inflaton is coupled to gauge fields, (p)reheating would primarily produce these massless vector fields~\cite{Adshead:2015pva}. If the number of gauge fields $\mathcal{N}$ is larger than one\footnote{Or if $\mathcal{N}=1$ but the related gauge field is not the Standard Model hypercharge.}, this could lead to a tension with the constraints on the amount of dark radiation present in the universe at the time of Big Bang Nucleosynthesis and of recombination~\cite{Ade:2015xua} if the gauge fields remain massless. Hence, the inclusion of the Standard Model degrees of freedom and of the (p)reheating should be consistent with these bounds. A detailed study of the (p)reheating process could also provide interesting clues for the generation of the baryon asymmetry of the universe, possibly sourced by the CP-violating axion-gauge field coupling~\cite{Kusenko:2014uta,Anber:2015yca,Adshead:2015jza}.

The high sensitivity of inflation to quantum corrections makes it crucial to embed any effective model into an UV complete theory. As remarked in App.~\ref{app:axions}, string theory contains all the ingredients necessary for the realization of an explicit embedding of the PBH production mechanism described in this paper: inflaton-dependent non-canonical kinetic terms for the axion-inflaton, a potential that supports axion-inflation and a coupling between the axion-inflaton and topological terms of Abelian gauge theories. The realization of an explicit string theory construction is beyond the scope of this paper, and we leave it for future work.

In Sec.~\ref{sec:PBHs} we have only reported a few benchmark models in order to show that the mechanism is well-suited for the production of PBHs. It would be interesting to perform a systematic analysis in order to understand which choices of the scalar potential and of the non-minimal coupling to gravity lead to the production of PBHs that compose a sizable fraction of the dark matter present in the universe - or vice versa how to reconstruct these fundamental ingredients of the theory for a given PBH distribution.

Finally, it would be interesting to extend this inflationary model to the case in which the axion-inflaton is coupled to non-Abelian gauge fields, and to study how the PBH production would be modified in such a setup.