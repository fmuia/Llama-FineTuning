\subsection{Coupled gauge field - inflaton background \label{subsec:dynamical_background}}

Previously in this section, we took $\xi$ to be a constant, external parameter in the equation of motion for the homogeneous gauge field background. We now turn to the complete dynamical background evolution, including also the evolution of the homogeneous inflaton field $\phi(\tau)$ and hence the (slow) evolution of $\xi$. This leads to the coupled system of equations
\begin{align}
f^{\prime\prime}(\tau)+2e^{2}f^{3}(\tau)-e\frac{\alpha}{\Lambda}\,\phi^{\prime}f^{2}(\tau)\,\  & =0 \,, \label{eq:fbackground2} \\
\phi''(\tau)+2aH\phi'(\tau)+a^{2}V_{,\phi}(\phi)+\frac{3\alpha e}{\Lambda a^{2}}f^{2}(\tau)f'(\tau) & =0 \,.
\label{eq:phibackground}
\end{align}
In single-field slow-roll inflation, $\xi$ typically increases over the course of inflation. This slowly evolving value of $\xi$ slightly modifies some of the results of the previous subsections (e.g.\ the precise values for the range of phases which lead to the $c_2$-solution in Fig.~\ref{fig:u0-of-xi} may be shifted), but the overall picture remains valid. After inserting the $c_2$-solution, Eq.~\eqref{eq:phibackground}  can be expressed as
\begin{equation}
 \frac{\dot \phi}{H} + \frac{V_{,\phi}}{V} +\frac{ \alpha}{e^2 \Lambda} H^2 (c_2 \, \xi)^3 = 0\,,
 \label{eq:phibackground2}
\end{equation}
where we have neglected the slow-roll suppressed term $\ddot \phi$. Assuming that the last term is sub-dominant, $\dot \phi/H$, $V_{,\phi}/V$ and $\xi$ are all proportional to $\sqrt{\varepsilon}$, with $\varepsilon = \dot \phi^2/(2 H^2) \simeq (V'/V)^2/2$ denoting the first slow-roll parameter. Moreover, for a quadratic or cosine potential as is usually considered in axion inflation models, $H^2$ is proportional to $1/\varepsilon$.  In summary, the time-dependence of all terms in Eq.~\eqref{eq:phibackground} is governed by the square root of the first slow-roll parameter. In particular, if the last term is sub-dominant at any point in time (after the $c_2$-solution has been reached), it will always remain sub-dominant. For the parameter example of Sec.~\ref{sec:example}, we find precisely this situation.

We note that this is a different regime than the `magnetic drift' regime studied in Refs.~\cite{Adshead:2012kp,Dimastrogiovanni:2012ew,Adshead:2013qp,Adshead:2013nka}. There, the friction term was taken to be large compared to the Hubble friction, $\xi \alpha H/(e \Lambda)  \gg 1$. Also in this regime, there is a local attractor for the gauge field background which scales as $f(\tau) \sim 1/\tau$, with however a different constant of proportionality. Within the non-abelian regime, the difference in our results with respect to these earlier works on CNI, in particular concerning the stability of the scalar sector, can be traced back to the fact that we do not restrict our analysis to this magnetic drift regime.






