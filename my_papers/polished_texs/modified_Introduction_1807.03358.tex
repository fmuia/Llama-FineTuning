
\section{Introduction\label{sec:introduction}}
The paradigm of cosmic inflation, proposed to explain the puzzling homogeneity and flatness of the Hot Big Bang Universe~\cite{Guth:1980zm}, has been strikingly successful in predicting the anisotropies of the Cosmic Microwave Background (CMB), measured to great precision by the Planck satellite~\cite{Ade:2015lrj}. This paradigm however, leaves open questions. What guarantees the required flatness of the inflationary potential? How is the inflation sector coupled to the Standard Model (SM) of particle physics? The lack of observable predictions on far sub-horizon scales makes it very difficult to find satisfactory and testable answers to these questions. In this context, a special role is played by pseudo-scalar inflation models, in which the inflaton $\phi$ (the particle driving inflation) couples to the field strength tensor $F_{\mu \nu}$ of massless gauge fields through the derivative coupling $\phi F_{\mu \nu} \tilde F^{\mu \nu}$. This coupling is compatible with a shift-symmetry of the inflaton $\phi$ protecting the flatness if the inflationary potential, it provides an immediate way to couple the inflation sector to a gauge field sector (which could be the SM or a hidden sector) and it leads to distinctive signatures, including a strongly enhanced chiral gravitational wave background~\cite{Cook:2011hg,Dimastrogiovanni:2012ew,Adshead:2013qp}.

The phenomenology of these models, both for abelian and non-abelian gauge fields, has recently received a lot of interest. In both cases, the gauge field sector experiences a tachyonic instability during inflation, leading to an explosive particle production which impacts the predictions of inflation. 
For abelian gauge fields this instability is controlled by the inflaton velocity, implying large effects towards the end of inflation in single-field slow-roll inflation models whereas the CMB scales can be largely unaffected, see Ref.~\cite{Barnaby:2010vf} for an overview. The phenomenology of this model includes a strongly enhanced and non-Gaussian contribution to the scalar and tensor power spectra~\cite{Barnaby:2010vf,Barnaby:2011qe,Barnaby:2011vw,Shiraishi:2013kxa,Cook:2013xea}, which may lead to a distortion of the CMB black body spectrum~\cite{Meerburg:2012id}, primordial black hole (PBH) production~\cite{Linde:2012bt, Domcke:2017fix,Garcia-Bellido:2016dkw} and an enhanced chiral gravitational wave signal in the frequency band of LIGO and LISA~\cite{Cook:2011hg,Barnaby:2011qe,Barnaby:2011vw,Anber:2012du,Domcke:2016bkh,Bartolo:2016ami}. Furthermore, the effective friction induced by the gauge field allows for inflation on rather steep potentials~\cite{Anber:2009ua}. The interplay of the gauge fields with the production of charged fermions has been studied in~\cite{Domcke:2018eki} and validity of the perturbative analysis has been scrutinized in~\cite{Ferreira:2015omg, Peloso:2016gqs}. 
The coupling to non-abelian $\mathrm{SU}(2)$ gauge fields, dubbed chromo-natural inflation (CNI) in~\cite{Adshead:2012kp}, allows for inflationary solutions on steep potentials in the presence of a non-vanishing isotropic background gauge field configuration. An analysis of the perturbations~\cite{Dimastrogiovanni:2012st,Dimastrogiovanni:2012ew,Adshead:2013qp,Adshead:2013nka} revealed an enhanced tensor power spectrum, however to the degree of excluding the model as an explanation for the anisotropies in the CMB. The same conclusion holds in the regime where the scalar field can be integrated out~\cite{Namba:2013kia}, referred to as  gauge-flation~\cite{Maleknejad:2011jw,Maleknejad:2011sq} (see also~\cite{Adshead:2012qe,SheikhJabbari:2012qf}). 
Modifications to the original model can evade this conclusion by employing different inflation potentials~\cite{Caldwell:2017chz,DallAgata:2018ybl}, by enlarging the field content of the model~\cite{Dimastrogiovanni:2016fuu,McDonough:2018xzh} or by considering a spontaneously broken gauge symmetry~\cite{Adshead:2016omu}. 


In this paper we study the possibility of a dynamical emergence of CNI, under plausible assumptions that we will discuss in due course. In CNI, the gauge field background is assumed to be homogeneous, isotropic and have a sufficiently large vacuum expectation value, so that the background evolution of the inflaton is dominated by the gauge friction term.  We show how such an isotropic background may develop from the regular Bunch--Davies initial conditions in the far past, providing justification for what is commonly taken for granted in CNI.\footnote{We emphasize that the mechanism presented in this paper is not a definitive solution to the problem of generating a background in CNI models: our arguments are based on a separation of length scales whose validity varies throughout the parameter space and should be explicitly verified in a dedicated lattice simulation.} For small gauge field amplitudes, the non-abelian $\mathrm{SU}(2)$ dynamics reduce to three copies of an abelian gauge group. As the inflaton velocity increases over the course of inflation, the tachyonic enhancement of the gauge fields in the abelian regime triggers a classical, inherently non-abelian background evolution. In this background, only a single helicity component of the gauge field features a regime of tachyonic instability. Contrary to the abelian case, each Fourier mode experiences this instability only for a finite time interval. 
We provide analytical results which make only minimal assumptions about the values of the parameters involved.
For the explicit parameter example which we study numerically, we find the gauge friction term to be subdominant in the non-abelian regime, contrary to the usual assumption in CNI. We emphasize that the transition from an effectively abelian to a non-abelian regime is generic in single field axion inflation models, and naturally removes the tension of the original CNI model with the Planck data by delaying the enhancement of the tensor power spectrum to smaller scales. Moreover, this dynamical transition implies that the catastrophic instability in the scalar sector, arising in part of the parameter space as pointed out in~\cite{Dimastrogiovanni:2012ew}, is generically avoided.




Throughout most of the paper we restrict ourselves to the linearized system of perturbations (see also~\cite{Dimastrogiovanni:2012st,Dimastrogiovanni:2012ew,Adshead:2013qp,Adshead:2013nka}). We however point out the importance of higher-order contributions to the scalar perturbation sector, taking into account that two enhanced helicity 2 gauge field perturbation can source helicity 0 (i.e.\ scalar) modes. We estimate the impact of this on the scalar power spectrum, finding an enhancement which is exponentially sensitive to the inflaton velocity, similar to what was found in the abelian case~\cite{Linde:2012bt}.\footnote{While this paper was being finalized, Refs.~\cite{Dimastrogiovanni:2018xnn,Papageorgiou:2018rfx} appeared, which also study the effects of the nonlinear coupling between the helicity $2$ and helicity $0$ perturbations. We briefly comment on these completely independent results in Sec.~\ref{sec:example}, finding overall good agreement within the expected uncertainties.}


The remainder of this paper is organized as follows. We begin with an executive summary in Sec.~\ref{sec:ExecutiveSummary}, to help  guide the reader through the different points discussed in this paper, followed by an overview on our notation in Sec.~\ref{sec:notation}. In Sec.~\ref{sec:axioninflation}, we review some of the key results and equations of abelian and non-abelian axion inflation, setting the notation for the following sections. Sec.~\ref{sec:background} is dedicated to the study of the emerging non-trivial homogeneous isotropic gauge field background. In Sec.~\ref{sec:linearized} we study the linearized system of perturbations in a general homogeneous isotropic gauge field background. This is applied to a specific parameter example in Sec.~\ref{sec:example}, showing explicitly the transition from the abelian to the non-abelian regime. We compute the resulting scalar and tensor power spectrum, taking into account non-linear contributions. We conclude in Sec.~\ref{sec:conclusions}. Six appendices deal with the derivation of the linearized perturbation equations, including the gravitational modes not included in the main text (App.~\ref{app:fulleom}), the explicit gauge field basis used in our linearized analysis (App.~\ref{app:basis}) details on the computation of the non-linear contributions to the scalar power spectrum (App.~\ref{app:variance_computation}), technical details supporting the analysis of the gauge field background (App.~\ref{app:sec3}), mathematical properties of homogeneous isotropic gauge fields (App.~\ref{app:gaugefields}) and analytical approximations of the Whittaker function describing the enhanced perturbation mode of the non-abelian regime (App.~\ref{app:asymptotics}).



\subsection{Executive summary \label{sec:ExecutiveSummary}}

To help guide the reader through the different aspects of our analysis, we give a preview of our key equations and results in this section, skipping all technical details. These results will be derived in the subsequent sections.

Our main focus will lie on the linearized regime of $\mathrm{SU}(2)$ axion inflation. The pseudo-scalar (axion-like) inflaton $\phi$ is coupled to the field strength tensor of the $\mathrm{SU}(2)$ gauge fields through the derivative coupling $\phi F_{\mu \nu} \tilde F^{\mu \nu}$. In the linearized regime, the $\mathrm{SU}(2)$ gauge field\footnote{We adopt a common abuse of notation by referring to the gauge potential $A^a_\mu$ as the \emph{gauge field}.} $A_\mu^a$ can be decomposed into a homogeneous isotropic background $f(\tau)$ and perturbations $\delta A_\mu^a$:\footnote{See Sec.~\ref{sec:notation} for our index conventions.} 
\begin{equation}
 A_\mu^a(\tau, \vec x) = f(\tau) \delta^a_\mu + \delta A^a_\mu (\tau, \vec x) \,.
\end{equation}
The classical evolution of the background gauge field is governed by
\begin{equation}
 \frac{\textrm{d}^2}{\textrm{d} \tau^2}(e f) + 2 (e f)^3 - \frac{2 \xi}{(- \tau)} (e f)^2 = 0 \,,
\end{equation}
where $e$ denotes the $\mathrm{SU}(2)$ gauge coupling and $\xi$, encoding the velocity of the inflaton and defined in Eq.~\eqref{eq:rev_xi}, is typically taken to be ${\cal O}(1 - 10)$ during the last 60 e-folds of inflation. 

For a slowly evolving inflaton, $\xi \simeq const.$, the classical background evolution is focused around two attractor solutions\footnote{In Sec.~\ref{subsec:dynamical_background} and~\ref{subsec:gaugefluctuations} we comment on the difference between the background evolution studied in this paper and the ‘magnetic drift regime’ of~\cite{Adshead:2012kp,Dimastrogiovanni:2012ew, Adshead:2013qp,  Adshead:2013nka}.},
\begin{equation}
 e f(\tau) = c_i \, \xi (- \tau)^{-1} \quad \text{with  } c_0 = 0 \,, \quad c_2 = \tfrac{1}{2} (1 + \sqrt{1 - 4/\xi^2}) \,,
\end{equation}
where the latter is only possible for $\xi \geq 2$. 
Beyond this classical motion, the background is also sourced by the fluctuations $\delta A^a_\mu$. These dominate the background evolution around the $c_0$-solution, and eventually trigger the transitions from the $c_0$ to the $c_2$ solution. For details see Sec.~\ref{sec:background}.

Out of the six physical degrees of freedom of the gauge field, the most important is the helicity $+2$ mode $w_{+2}$, which couples directly to the metric tensor mode, sourcing chiral gravitational waves (see also~\cite{Dimastrogiovanni:2012ew,Adshead:2013qp}). In the $c_2$ background solution, its equation of motion 
\begin{equation}
 \frac{\textrm{d}^2}{ \textrm{d} x^2} w_{+2}(x) + \left(1 - \frac{2 \xi}{x} + 2 \left(\frac{\xi}{x} - 1 \right) \frac{c_2 \xi}{x} \right) w_{+2}(x) = 0 \,,
 \label{eq:eom2Intro}
\end{equation}
(where $x = - k \tau$ with $k$ the momentum of the Fourier mode $w_{+2}$) has an exact solution in terms of the Whittaker function in the limit of constant $\xi$:
\begin{equation}
w_{+2}^{(e)}(x)=e^{(1+c_{2})\pi\xi/2}W_{-i \kappa,\,-i\mu}\left(-2ix\right) \,,
\label{eq:AnalyticalSolutionIntro}
\end{equation}
with $\kappa = (1 + c_2) \xi$ and  $\mu = \xi \sqrt{2 c_2 - (2 \xi)^{-2}}$. Due to a tachyonic instability in Eq.~\eqref{eq:eom2Intro} in between $x_\text{max,min} = (1 + c_2 \pm \sqrt{1 + c_2^2})\xi$, this solution is strongly enhanced just before horizon crossing. At and after horizon crossing, Eq.~\eqref{eq:AnalyticalSolutionIntro} is well approximated by
\begin{equation}
  \, w_{+2}(x) \simeq 2 e^{(\kappa - \mu) \pi} \sqrt{\frac{ x}{ \mu}} \cos\left[\mu \ln(2 x) + \theta_0 \right] \,.
\end{equation}
With this solution at hand, we can approximately analytically solve the 
coupled system of helicity $+2$ gauge fields and gravitational waves 
(see Eq.~\eqref{eq:+2mode}), obtaining for the gravitational wave amplitude $w_{+2}^{(\gamma)}$ after freeze-out on super-horizon scales
\begin{equation}
 x \, w_{+2}^{(\gamma)}(x) \big|_{x \geq 1} \simeq -  \frac{2 H \xi^{5/2}}{e } 2^{3/4} e^{(2 - \sqrt{2}) \pi \xi} \,,
\end{equation}
and consequently for the amplitude of the chiral stochastic gravitational wave background (see Eq.~\eqref{eq:GWapprox} for details),
\begin{equation}
 \Omega_{\rm GW} \simeq \frac{1}{24} \Omega_r \left( \frac{\xi^3 H}{\pi M_P} \right)^2_{\xi = \xi_\text{cr}}  \left( \frac{2^{7/4} H}{e} \xi^{-1/2} e^{(2 - \sqrt{2}) \pi \xi}  \right)^2_{\xi = \xi_\text{ref}} \,.
\end{equation}

The scalar perturbations are not enhanced at the linear level in the parameter space in the focus of this work. However, non-linear contributions, sourced by two enhanced helicity $+2$ gauge field modes, yield an exponentially enhanced contribution to the scalar power spectrum. We report analytical estimates for the resulting contribution to the scalar power spectrum in Eq.~\eqref{eq:Ds_2nd}. 

Combining the results on the background evolution and the analysis of the perturbations, the following picture emerges: At early times, deep in de-Sitter space with small values of $\xi$, the non-abelian axion inflation model reduces to the abelian regime. Two factors are necessary to trigger the transition to the inherently non-abelian regime: The $c_2$ solution of the classical background emerges at $\xi \geq 2$ and the gauge field fluctuations have to reach a sufficient amplitude to trigger initial conditions for the classical motion which actually lead to the $c_2$ solution. We emphasize that the linearized description of this transition is based on two assumptions, which we will justify in Secs.~\ref{sec:axioninflation} and \ref{sec:linearized}, respectively: (i) the gauge fields sourced in the abelian regime are approximately homogeneous over a Hubble patch and (ii) the gauge field fluctuations in the non-abelian regime are small compared to this homogeneous background.\footnote{A quantification of the resulting uncertainties on our final results most likely requires a lattice simulation of the full non-linear theory in de Sitter space. Current state-of-the-art techniques~\cite{Adshead:2015pva,Cuissa:2018oiw} can however only evolve this system for a few Hubble times, insufficient to address this question. We hope that this work will trigger future research in this direction. }

As a proof of concept, we study a parameter example in Sec.~\ref{sec:example} in which the CMB scales exit the horizon in the abelian regime at relatively small $\xi$ (thus ensuring agreement with all CMB observations), whereas smaller scales exit the horizon after the transition to the inherently non-abelian regime. The resulting scalar and tensor power spectra are strongly enhanced at small scales, see Fig.~\ref{fig:spectraPS} and \ref{fig:spectraGW}.

\subsection{Notation and conventions \label{sec:notation}}
We summarize here the main conventions used throughout this paper. The metric signature is $(-,+,+,+)$ and we mostly employ conformal time $\tau$ instead of cosmic time $t$. Derivatives with respect to the conformal time are denoted by a prime, while derivatives with respect to the cosmic time are denoted by a dot. We often use the dimensionless variable
\begin{equation}
x = - k \tau \,.
\end{equation}
The first (second) derivative of the functional 
$\mathcal{S}(\phi)$
with respect to the field $\phi$ is denoted by 
$\mathcal{S}_{, \phi}$ ($\mathcal{S}_{, \phi \phi}$). 
The Fourier transform of the function  
$F(t,\vec{x})$
(or similarly for
$F(\tau,\vec{x})$)
is given by
\begin{equation}
F(t,\vec{x}) = \int \frac{\textrm{d}^3 \vec{k}}{\left(2 \pi\right)^{3/2}} \, \tilde{F}(t,\vec{k}) e^{-i \vec{k} \cdot \vec{x}} \,.
\end{equation}
(Anti-)Symmetrization is defined as
\begin{equation}
S_{(ij)} = \frac{S_{ij} + S_{ji}}{2} \,, \qquad A_{[ij]} = \frac{A_{ij} - A_{ji}}{2} \,.
\end{equation}
Greek letters refer to space-time indices ($\mu = 0, 1, 2, 3$), roman letters from the beginning of the alphabet refer to gauge indices (e.g. $a = 1, 2, 3$ for a $\mathrm{SU}(2)$ gauge group) and roman letters from the middle of the alphabet refer to spatial indices ($i = 1, 2, 3$). We use the usual conventions for $\mathrm{SU}(N)$ gauge fields $\mathbf{A}_\mu = A^a_\mu \mathbf{T}_a$. The field strength tensor is defined as 
\begin{equation}
	F^a_{\mu\nu} \mathbf{T}_a \equiv \mathbf{F}_{\mu\nu} \equiv \frac{i}{e}[\mathbf{D}_{\mu},\mathbf{D}_{\nu} ] = \partial_\mu \mathbf{A}_{\nu} - \partial_\nu \mathbf{A}_{\mu} - i e[\mathbf{A}_{\mu},\mathbf{A}_{\nu}]  \; ,
\end{equation}
where $e$ is the coupling constant, $\mathbf{T}_a$ is the $a$-th generator of the group, $\mathbf{A}_{\mu} \equiv A^a_{\mu} \mathbf{T}_a$ and where we have used the definition of covariant derivative:
\begin{equation}
	\mathbf{D}_{\mu} \equiv \partial_{\mu} - i e \mathbf{A}_{\mu} \equiv \partial_{\mu} - i e A^a_{\mu} \mathbf{T}_a\; .
\end{equation} 
With the commutation relation
\begin{equation}
	 [\mathbf{T}_a, \mathbf{T}_b] = i \varepsilon_{abc}  \mathbf{T}_c \; ,
\end{equation}
 the field strength can be expressed as
\begin{equation}
	\label{eq:field_strength}
	F^a_{\mu\nu} = \partial_\mu A^a_{\nu} - \partial_\nu A^a_{\mu} + e \varepsilon^{abc} A^b_{\mu}A^c_{\nu} \; .
\end{equation}
The dual tensor to the field strength is defined as
\begin{equation}
\tilde{F}_a^{\mu \nu} = \frac{\varepsilon^{\mu \nu \rho \sigma}}{2 \sqrt{-g}} F^a_{\rho \sigma} \,,
\end{equation}
where we use the convention $\varepsilon^{0123} = 1$ for the anti-symmetric tensor. Additional conventions related to the computation of the equations of motion in the ADM formalism are reported in App.~\ref{app:fulleom}.

