The recent detection of gravitational waves (GWs) emitted by a black hole (BH) binary by the LIGO/VIRGO collaboration~\cite{Abbott:2016blz} marked not only the beginning of a new age of GW astronomy but can also be seen as the first direct evidence of BHs. With a lot more data from advanced LIGO/VIRGO and from the future space-mission LISA~\cite{LISA} on the horizon, these are exciting times for considering possible implications for fundamental physics and early universe cosmology. 

Prompted by the LIGO discovery and the lack of positive results in dedicated dark matter searches, the idea that primordial black holes (PBHs) could be (a fraction of) dark matter has recently received a lot of attention~\cite{Bird:2016dcv, Clesse:2016vqa,Sasaki:2016jop,Carr:2016drx,Clesse:2016ajp,Blinnikov:2016bxu,Garcia-Bellido:2017fdg,Georg:2017mqk}. PBHs are formed when local over densities collapse due to their gravitational instability. Here we will focus on the possibility that these local over densities are sourced by the primordial scalar power spectrum of inflation (see e.g.\ Refs.~\cite{Cotner:2016cvr,Davoudiasl:2016mwf } for other recently discussed possibilities). This requires an amplitude of the scalar power spectrum which is much larger than the amplitude observed in the cosmic microwave background (CMB), i.e.\ a highly blue spectrum. On the other hand, PBHs formed at very small scales are strongly constrained by the traces their decays leave in the CMB and in the extra-galactic gamma ray background~\cite{Carr:2009jm}. A viable model of PBH dark matter thus requires a non-trivial peaked structure of the primordial scalar power spectrum. It has been demonstrated that such spectra can be achieved in multifield models of inflation~\cite{GarciaBellido:1996qt,Lyth:2011kj,Bugaev:2011wy,Clesse:2015wea,Kohri:2012yw,Kawasaki:2012wr,Bugaev:2013vba,Inomata:2017okj,Inomata:2016rbd,Kawasaki:2016pql} or by arranging for the corresponding features in the scalar potential of inflation~\cite{Cheng:2016qzb,Garcia-Bellido:2017mdw,Garcia-Bellido:2016dkw}.

In this paper, we generate PBH dark matter in an inflation model which is based on two simple, generic features: an axion-like inflaton with a non-minimal coupling to gravity which increases towards the end of inflation. Axion-like particles are interesting inflaton candidates for several reasons: (i)~their approximate shift-symmetry protects them from large mass contributions, preserving the required slow-roll phase of inflation. (ii)~they appear abundantly in supergravity as part of complex scalars as well as in string theory~\cite{0605206}, where they typically arise upon compactification of form field gauge potentials. (iii)~they generically couple to the topological Chern-Simons term of any gauge group, leading to a wide range of potentially observable effects~\cite{Turner:1987bw, Garretson:1992vt, Anber:2006xt,Barnaby:2011qe}. These include a non-Gaussian and blue contribution to the scalar spectrum, potentially leading to PBH formation~\cite{Barnaby:2010vf,Linde:2012bt,Lin:2012gs,Meerburg:2012id,Bugaev:2013fya,Garcia-Bellido:2016dkw} as well as a blue, non-Gaussian and highly polarized stochastic GW spectrum, potentially observable by LIGO or LISA~\cite{Cook:2011hg,Anber:2012du,Bartolo:2016ami}.
%
The predictions for different universality classes of pseudoscalar inflation have recently been studied in Ref.~\cite{Domcke:2016bkh}. Here we extend this work by allowing for a non-minimal coupling to gravity, i.e.\ by considering simple Jordan frame inflation models. In recent years, such a setup was found to be appealing both for deriving the supergravity Lagrangian~\cite{Kallosh:2000ve} and from a phenomenological perspective, driven most notably by the discussion of Higgs inflation~\cite{Bezrukov:2007ep}. Such a non-minimal coupling to gravity is a characteristic feature of supergravity in the Jordan frame~\cite{Kallosh:2013hoa,Kallosh:2013tua,Ferrara:2013rsa,Galante:2014ifa,Broy:2015qna,Das:2016kwz}. The so-called attractor models at strong coupling~\cite{Kallosh:2013tua,Kallosh:2013yoa} are an example of how the non-minimal coupling to gravity can be used to construct classes of models which asymptote to the sweet spot of the Planck data, see also~\cite{Einhorn:2009bh,Ferrara:2010yw,Buchmuller:2013zfa,Giudice:2014toa,Pallis:2013yda,Pallis:2014dma,Pallis:2014boa,Ellis:2013xoa,Kallosh:2013xya,Nakayama:2010ga,Pieroni:2015cma} for other implementations.


Combining these two ingredients, the following picture emerges: at large scales (constrained by CMB observations~\cite{Ade:2015lrj}), i.e.\ early on in the inflationary epoch, both the coupling to the Chern-Simons term and the non-minimal coupling to gravity are suppressed. As inflation proceeds, the former generically induces a tachyonic instability in the equation of motion for the gauge fields, leading to a gauge-field background which in turn acts as an additional, classical source term for scalar and tensor perturbations. This results in a strong increase of both spectra. Once the gauge field production has reached a critical value, the back-reaction of the gauge fields on the inflaton dynamics slows the growth of both spectra~\cite{Anber:2009ua,Barnaby:2011qe,Barnaby:2011vw}. Finally, in the last stage of inflation the non-minimal coupling to gravity becomes relevant and both the scalar and tensor spectrum are suppressed. As a result, a significant fraction of dark matter (for specific parameter choices even all of dark matter) can be contained in PBHs. In part of the parameter space their mergers may be observable by advanced LIGO. We observe a remarkable complementarity between CMB observations, PBH constraints and GW searches in constraining the parameter space of this setup.


The remainder of this paper is organized as follows. After a review of axion inflation in Sec.~\ref{sec:review}, we extend the formalism to account for a non-minimal coupling to gravity in Sec.~\ref{sec:non-minimal}. This in particular includes the computation of the resulting scalar and tensor perturbations, sourced by both the quantum mechanical vacuum contribution and the classical gauge field contribution. In Sec.~\ref{sec:attractors} we apply these results to inflation models described by attractors at strong coupling. We dedicate Sec.~\ref{sec:PBHs} to the production of primordial black holes, discussing different regions of the parameter space based on a handful of benchmark models. We conclude in Sec.~\ref{sec:conclusions}. The main results of this paper are supported by three appendices. In App.~\ref{sec:axions} we discuss how the type of axions we consider here arise in the context of field theory, supergravity and string theory. Some of the details of the derivations of the scalar power spectrum are left to App.~\ref{sec:spectrum}. Finally, App.~\ref{app:PBH} analyses how uncertainties related to the PBH formation and evolution impact our results.